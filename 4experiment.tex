\section{Experimental Results}\label{sec:4experiment}

\subsection{Evaluation of Subcomponents}
In this section, we evaluate the accuracy of \systemname on detecting five sleep events: the body posture, the body rollover, the hand
position, micro body movements and acoustic events. The ground truth of an event is obtained by watching or listening to the video footage.
We use \emph{cross-validation} to train and evaluate our algorithms. Specifically, we train our algorithms using the data collected from
one participant and apply the learned algorithms to the remaining 14 participants. We repeat this process by using each participant's data
as the training data in turn. We then report the the average performance across the cross-validation folds. This is a standard evaluation
methodology, providing an estimate of the generalization ability of a predictive model when processing \emph{unseen} data.

\subsubsection{Sleep Posture Detection}
\label{subsub:bodyposture} Fig.~\ref{fig:posture_zhu} reports the sleep posture detection accuracy for individual users. \systemname
achieves a consistent detection accuracy of over 90\% across testing users and postures. The confusion matrix in Table \ref{tab:posture}
shows that \systemname only misclassifies a posture on a few occasions, leading to an overall detection precision of over 96\% across
users. This performance is better than the one reported by SleepMonitor~\cite{sleepmonitor}. In particular, {\systemname} gives an
improvement of 5\% for the prone state and has an overall lower false positive rate. Furthermore, {\systemname} is able to detect more hand
positions during sleeping when compared to SleepMonitor. In terms of errors, due to angular characteristics of acceleration being similar
between the supine posture where the hand is put on the head and the left-lateral posture, a small amount of the supine postures are
misclassified as the left lateral. From the results, we can also observe that the total number of detected prone postures is smaller than
that for other postures. This suggests that our testing users are not used to sleep in this position, because it is neither healthy nor
comfortable.






\begin{figure}
	\centering
	\begin{minipage}{.5\textwidth}
	 \centering
	\includegraphics[width=0.95\linewidth]{Figures/posture_zhu.pdf}
	\caption{Detection accuracy of body postures.}\label{fig:posture_zhu}
	\end{minipage}%
	\begin{minipage}{.5\textwidth}
			\centering
		\includegraphics[width=0.95\linewidth]{Figures/handposition_zhu.pdf}
		\caption{Identification accuracy of hand positions.}\label{fig:hand_zhu}
	\end{minipage}
\end{figure}

\begin{table}[!t]\footnotesize
	\centering
	\renewcommand\arraystretch{0.35}
	\caption{The confusion matrix of body posture classification.}\label{tab:posture}
	\begin{tabular}{c| r | r | r | r | r | r}
		\cline{1-7}
		&\multicolumn{1}{ c|}{ }
		& \multicolumn{4}{ c|}{ }\\
		\multirow{2}*{}
		&\multicolumn{1}{c|}{\multirow{2}*{{\textbf{Result}}}}
		&\multicolumn{4}{c|}{{\textbf{Prediction}}}
		& \multirow{4}*{{\textbf{Recall}}} \\
		\cline{3-6}
		& & & & & \\
		\multicolumn{1}{c|}{{}}
		&  \multicolumn{1}{c|}{{}}
		&  \multicolumn{1}{l|}{{Supine}}
		&  \multicolumn{1}{l|}{{Left Lateral}}
		&  \multicolumn{1}{l|}{{Right Lateral}}
		&  \multicolumn{1}{l|}{{Prone}}   \\
		& & & & & \\
		\cline{1-7}
		& & & & & \\
		\multirow{5}{*}{\begin{sideways}{{Groundtruth}}\end{sideways}}
		&   {Supine}   & {\bf{{1182}}}    &   $25$      &   $4$      &   $9$    &   {96.7\%}\\
		& & & & & \\
		\cline{2-7}
		& & & & & \\
		&   {Left Lateral}   &   $6$      &   {\bf{{1292}}}     &   $0$      &   $0$   &   {99.5\%} \\
		& & & & & \\
		\cline{2-7}
		& & & & & \\
		&   {Right Lateral}   &   $7$      &   $0$      &  {\bf{{1275}}}      &   $12$  &   {98.5\%}  \\
		& & & & & \\
		\cline{2-7}
		& & & & & \\
		&   {Prone}   &   $19$      &   $2$      &   $3$      &   {\bf{{567}}}   &   {95.9\%} \\
		& & & & & \\
		\cline{1-7}
		& & & & & \\
		&   {Precision}    &   {97.3 \%}   &   {98.0\%}   &   {99.5\%}   &   {96.4\%}    \\
		& & & & & \\
		\cline{1-7}
	\end{tabular}
\end{table}


\subsubsection{Body Rollover Counting}
Table \ref{tab:rollver} reports the performance of body rollover counting. Three of our participants, users 3, 4 and 13, have an unusually
high number of rollovers. For users 3 and 4, they have difficulties in falling asleep due to the sleep disorder, while user 13 needs to
rollover frequently because of his loudly snoring. As we will demonstrate in Sec.~\ref{sec:user_survey}, these participants also suffered
from poor sleep quality and hence indicate how the information extracted by {\systemname} can support the detection of sleep problems. For
all the 15 users, the detection accuracies are all high with the lowest accuracy of 87\%. Therefore, {\systemname} can accurately
distinguish the large hand movement from the body rollover. We stress the the small detection errors in body rollover events do not have a
significant impact on our end result. This is because for sleep stage detection, we consider a multiple events in addition to body
rollover, including micro body movement and acoustic events, which can offset the detection errors for body rollover.

\begin{table}[!t]\footnotesize
  \caption{Detection accuracy of body rollover.}\label{tab:rollver}
  \setlength{\tabcolsep}{3pt}
\renewcommand{\arraystretch}{0.67}{\multirowsetup}{\centering}
        \begin{tabular}{lccccccccccccccc}
        \toprule
         \textbf{Testing User ID}    & 1& 2  & 3& 4& 5& 6& 7& 8& 9& 10& 11& 12& 13& 14& 15\\
        \midrule
            \rowcolor{Gray} {Labeled \#body-rollover}  &231&204&442&397&198&101&196&164&193&208&131&205&342&149&156 \\
                 {Accuracy} &91\%& 94\% &88\%&93\%&96\%&94\%&87\%&90\% &93\% &94\% &92\% &94\% &89\% &90\% &95\%\\
        \bottomrule
 \end{tabular}
\end{table}

\subsubsection{Hand Position Recognition}
Recall that we combine the tracked hand movement trajectory and periodic signals caused by respiration to identify if the hand is placed on
the chest, addomen or head. In our dataset, 14\%, 36\% and 22\% of the time the hand in the supine posture during sleep were placed on the
head, abdomen, and chest respectively. Fig.~\ref{fig:hand_zhu} illustrates the accuracy of hand position across 15 users. As we can see
that using just one single user's data for training, our approach achieves an accuracy of over 87\% across different testing users.
Moreover, we find that at least 4 out of our 15 participants tended to put their hands on their heads (which often lead to bad sleep) and
one participate unconsciously put his hand on the chest (which is likely to cause nightmare). By identifying these hand positions,
\systemname can inform users to change their sleeping behaviors.

\subsubsection{Micro Body Movement Detection}

In this experiments, in addition to manually labeling the ground truth from the video footage, we also use the accelerometer in the
smartphone placed on the bed to record the occurrence of micro body movements. This allows us to avoid missing some movements such as
trembling concealed by the duvet (which may not be seen from the video footage). For the smartphone data, we first smooth the acceleration
along the three axes. Next, we calculate the Root Sum Square (RSS) to merge the results across axes to obtain the first-order derivative of
the merged acceleration. Then, we use a threshold-based detection method to mark the occurrence of motions from the smartphone data.
Table~\ref{tab:micro_move} lists the total number of the micro body movements for each participant over the testing period of 14 days, and
Fig.~\ref{fig:micro_movement_zhu} reports the accuracy of {\systemname} for detecting these micro body movements. As can be seen from the
table and figure, \systemname achieves consistent precision across our testing users. Figure~\ref{fig:micro_combine} reports the averaged
accuracy across users. The averaged accuracies for arm raising and body trembling are higher, 93\% and 84\% respectively.  While we achieve
the lowest accuracy for detecting hand movements, the average procession and recall still exceed 75\%. The accuracy for detecting hand
movements and body trembling can be further improved by using more training data collected from a longer period of time. It is to note that
the purpose of micro body movement detection is to detect different sleep stages. Since hand movements usually appear in all sleep stages,
they offer fewer information gains for identifying sleep stages compared to other movements.  As a result, the relatively low detection
accuracy of hand moving has little impact on our final result.


\begin{figure*}
	\centering
	\begin{minipage}{.485\textwidth}
		 \includegraphics[width=0.95\textwidth]{Figures/micro_movement_zhu.pdf}
		\caption{Micro body movement detection accuracy for each user.}\label{fig:micro_movement_zhu}	
	\end{minipage}%
\hspace{3pt}
	\begin{minipage}{.485\textwidth}
	 \centering
	\includegraphics[width=7.8cm,height=5cm]{Figures/micro_combine1.pdf}
	\caption{Average precision and recall for micro body movement detection.}\label{fig:micro_combine}
	\end{minipage}
\end{figure*}


\begin{table}[!t]\footnotesize
  \caption{The number of micro body movements per user.}\label{tab:micro_move}
   \renewcommand\arraystretch{1}{\multirowsetup}{\centering}
        \begin{tabular}{lccccccccccccccc}
        \toprule
         \textbf{Testing User ID}    & 1& 2  & 3& 4& 5& 6& 7& 8& 9& 10& 11& 12& 13& 14& 15\\
        \midrule
            \rowcolor{Gray} {Labeled \#hand movement}  &52&49&67&55&78&65&59&70&61&53&81&55&60&59&63 \\
             { Labeled \#arm raising} &48&50&62&53&66&49&57&50&73&45&54&69&57&56&61\\
             \rowcolor{Gray} { Labeled \#body trembling} &28&32&25&29&34&25&20&30&24&26&27&35&24&22&25\\
        \bottomrule
 \end{tabular}
\end{table}


 \subsubsection{Acoustic Events Detection}
 Table~\ref{tab:sound} shows the results for acoustic event detection across 15 participants. Overall, \systemname is highly accurate in
 detecting acoustic events, with an average accuracy of over 88\%. The precision for detecting coughs is
88.9\%, which is slightly lower than for the other three event types. The reason is that different users have distinct cough patterns but
the pre-defined parameters in the detection model do not include all patterns. For examples, some people have a fast and continuous pattern
of coughing, while others have a slower intermittent pattern. Our model for cough detection is trained on 120 sets of nighttime sound data
collected from users who are different from our testing users. To further improve accuracy, we can either specialize our model using data
collected from the target user or using a larger and more diverse set of training data. Nonetheless, \systemname is able to detect acoustic
events with a high accuracy.


\begin{table}[!t]\footnotesize
  \centering
 \renewcommand\arraystretch{0.35}
  \caption{The confusion matrix of acoustic events detection.}\label{tab:sound}
\begin{tabular}{c| r | r | r | r | r | r}
   \hline
   &\multicolumn{1}{ c|}{ }
   & \multicolumn{4}{ c|}{ }\\
   \multirow{2}*{}
&\multicolumn{1}{c|}{\multirow{2}*{{ \textbf{Result}}}}
&\multicolumn{4}{c|}{{ \textbf{Prediction}}}
& \multirow{4}*{{ \textbf{Recall}}} \\
    \cline{3-6}
    & & & & & \\
    \multicolumn{1}{c|}{{}}
    &  \multicolumn{1}{c|}{{}}
    &  \multicolumn{1}{l|}{{ Snore}}
    &  \multicolumn{1}{l|}{{ Cough}}
    &  \multicolumn{1}{l|}{{ Somniloquy}}
    &  \multicolumn{1}{l|}{{ Other}}   \\
    & & & & & \\
     \cline{1-7}
    & & & & & \\
    \multirow{5}{*}{\begin{sideways}{{ Groundtruth}}\end{sideways}}
    &   { Snore}   & {\bf{{96}}}    &   $0$      &   $0$      &   $9$    &   {91.4\%}\\
    & & & & & \\
    \cline{2-7}
    & & & & & \\
   &   { Cough}   &   $3$      &   {\bf{{64}}}     &   $0$      &   $4$   &   {90.1\%} \\
    & & & & & \\
     \cline{2-7}
    & & & & & \\
    &   { Somniloquy}   &   $0$      &   $3$      &  {\bf{{42}}}      &   $2$  &   {89.4\%}  \\
    & & & & & \\
     \cline{2-7}
    & & & & & \\
    &   { Other}   &   $0$      &   $5$      &   $4$      &   {\bf{{325}}}   &   {97.3\%} \\
    & & & & & \\
    \hline
    & & & & & \\
    &   { Precision}      &   {96.9\%}   &   {88.9\%}   &   {91.3\%}   &   {95.6\%}    \\
    & & & & & \\
    \hline
   \end{tabular}
\end{table}


\subsection{Overall Performance\label{sec:overall_per}}

\subsubsection{Sleep Stage Detection}

In order to prove that the detected events not only reflect the user's sleep habits, but also effectively identify the sleep stages to assess the sleep quality, we regard the reported results from Fitbit Charge2 as the ground truth. To perform the evaluation, we randomly choose $50$ sets of sleep data from the data so that at least $3$ sets per participant are chosen for evaluation. For detecting changes in sleep stage, {\systemname} uses event-driven detection. When there is no sleep event detected in 15 minutes, we evaluate the sleep stage. When an event occurs, we immediately evaluate the sleep stage and use this time as the starting point for the next 15 minutes. The averaged precision value and recall value are shown in Table~\ref{tab:sleep stage}. It indicates that though {\systemname} may make misjudgment between the light sleep and REM, overall the performance is satisfying and comparable to current consumer-grade monitors. Moreover, as we later demonstrate, the main benefits of {\systemname} result from its capability to estimate a wide range of sleep events and how they relate to sleep quality, not from its performance in sleep stage detection where medical PSG measurements are required for accurate assessment of sleep stages.

\begin{table}[!t]\footnotesize
%\setlength{\tabcolsep}{1pt}
\renewcommand{\arraystretch}{0.55}{\centering}
	\caption{{The confusion matrix of sleep stage detection.}}\label{tab:sleep stage}
	\begin{tabular}{c| r | r | r | r | r}
		\hline
		&\multicolumn{1}{ c|}{ }
		& \multicolumn{3}{ c|}{ }\\
		\multirow{2}*{}
		&\multicolumn{1}{c|}{\multirow{2}*{{ \textbf{Result}}}}
		&\multicolumn{3}{c|}{{ \textbf{Prediction}}}
		& \multirow{3}*{{ \textbf{Recall}}} \\
		%&\multicolumn{5}{ c |}{\textbf{\small Prediction}} \\
		% & \multicolumn{5}{ c |}{ } \\
		\cline{3-5}
		& & & & & \\
		\multicolumn{1}{c|}{{}}
		&  \multicolumn{1}{c|}{{}}
		&  \multicolumn{1}{l|}{{ REM}}
		&  \multicolumn{1}{l|}{{ Light Sleep}}
		&  \multicolumn{1}{l|}{{ Deep Sleep}} \\
		\cline{1-6}
		& & & & & \\
		\multirow{1}{*}{\begin{sideways}{{Groundtruth}}\end{sideways}}
		&   { REM}   & {\bf{{476}}}    &   $143$      &   $61$     &   {70.0\%}\\
		& & & & & \\
		\cline{2-6}
		& & & & & \\
		&   { Light Sleep}   &   $131$      &   {\bf{{508}}}     &   $91$      &   {69.6\%} \\
		& & & & & \\
		\cline{2-6}
		& & & & & \\
		&   { Deep Sleep}   &   $63$      &   $113$      &  {\bf{{262}}}      &   {59.8\%}  \\
		& & & & & \\
		\cline{1-6}
		& & & & & \\
		&   { Precision}      &   {71.0\%}   &   {66.5\%}   &   {63.3\%}   \\
		& & & & & \\
		\hline
	\end{tabular}
\end{table}

\subsubsection{Effect of Respiratory Amplitude on Sleep Stage Detection}

When we detect different sleep stages, we also consider the respiratory amplitude when the hand's position is in the abdomen or chest. To assess the effectiveness of respiratory amplitude estimation, we evaluate the performance of the sleep stage detection in two cases, that is with and without taking the respiration amplitude into account. The performance of sleep stage detection is shown in Table~\ref{tab:respiratory}. For three different sleep stages, both the precision and recall values are improved with the help of respiratory amplitude estimation. In fact, the respiratory frequency can also be used as a feature to help us to detect sleep stages. But in fact their essence the same. The difference in respiratory amplitude will also affect the difference in respiratory frequency, because when the respiratory amplitude is large, the time taken for one breath will be long, and the frequency of breathing will be slower. In {\systemname}, we choose the respiratory amplitude because the feature is very intuitive.

\begin{table}[!t]\footnotesize
	\centering
	\renewcommand\arraystretch{0.3}
	\caption{Effect of respiratory amplitude estimation.}\label{tab:respiratory}
	\begin{tabular}{l| l | r | r | r | r | r | r |}
		\cline{2-8}
		&\multicolumn{1}{ c|}{ }
		&\multicolumn{2}{ c|}{ }
		&\multicolumn{2}{ c|}{ }
		& \multicolumn{2}{ c|}{ }\\
		%  \multirow{4}*{}
		&\multicolumn{1}{c|}{}
		&\multicolumn{2}{c|}{\textbf{\footnotesize REM}}
		&\multicolumn{2}{c|}{\textbf{\footnotesize Light Sleep}}
		&\multicolumn{2}{c|}{\textbf{\footnotesize Deep Sleep}} \\
		%&\multicolumn{5}{ c |}{\textbf{\small Prediction}} \\
		% & \multicolumn{5}{ c |}{ } \\
		\cline{2-8}
		& & & & & & &\\
		\multicolumn{1}{c|}{\textbf{}}
		&  \multicolumn{1}{l|}{\textbf{Features}}
		&  \multicolumn{1}{l|}{\footnotesize Precision}
		&  \multicolumn{1}{c|}{\footnotesize Recall}
		&  \multicolumn{1}{c|}{\footnotesize Precision}
		&  \multicolumn{1}{c|}{\footnotesize Recall}
		&  \multicolumn{1}{c|}{\footnotesize Precision}
		&  \multicolumn{1}{c|}{\footnotesize Recall}\\
		& & & & & & &\\
		\cline{2-8}
		& & & & & & &\\
		\multirow{5}{*}
		&   \textbf{\footnotesize Without Respiratory Amplitude}   & $62.9\%$    &   $63.4\%$      &   $59.4\%$      &   $63.9\%$    &   $57.7\%$ &  $54.1\%$ \\
		& & & & & & &\\
		\cline{2-8}
		& & & & & & &\\
		&   \textbf{\footnotesize With Respiratory Amplitude}   &   $71.0\%$      &   $70.0\%$     &   $66.5\%$      &   $69.7\%$   &   $63.3\%$ &   $59.8\%$ \\
		& & & & & & &\\
		\cline{2-8}
	\end{tabular}
\end{table}

  \begin{table}[!t]\footnotesize
 	\centering
 	\renewcommand\arraystretch{0.3}
 	\caption{Performance of sleep stage detection comparison.}\label{tab:comparison}
 	\begin{tabular}{c| r | r | r | r | r |}
 		\cline{2-6}
 		&\multicolumn{1}{ c|}{ }
 		&\multicolumn{2}{ c|}{ }
 		&\multicolumn{2}{ c|}{ }\\
 		&\multicolumn{1}{c|}{}
 		&\multicolumn{2}{c|}{\textbf{\footnotesize Light Sleep}}
 		&\multicolumn{2}{c|}{\textbf{\footnotesize Deep Sleep}} \\
 		\cline{2-6}
 		\multicolumn{1}{c|}{\textbf{}}
 		&  \multicolumn{1}{l|}{\diagbox{System}{Stage}}
 		&  \multicolumn{1}{c|}{\footnotesize Precision}
 		&  \multicolumn{1}{c|}{\footnotesize Recall}
 		&  \multicolumn{1}{c|}{\footnotesize Precision}
 		&  \multicolumn{1}{c|}{\footnotesize Recall}\\
 		\cline{2-6}
 		& & & & & \\
 		&   \textbf{\footnotesize {\systemname}}   & $66.5\%$    &   $69.6\%$      &   $63.3\%$      &   $59.8\%$  \\
 		& & & & &  \\
 		\cline{2-6}
 		& & & & & \\
 		&   \textbf{\footnotesize Sleep As Android}   &   $27.8\%$      &   $35.4\%$     &   $35.7\%$      &   $50.2\%$   \\
 		& & & & &  \\
 		\cline{2-6}
 		& & & & & \\
 		&   \textbf{\footnotesize Sleep Hunter}   &   $66.7\%$      &   $66.1\%$     &   $60.0\%$      &   $50.7\%$   \\
 		& & & & &  \\
 		\cline{2-6}
 	\end{tabular}
 \end{table}

\subsubsection{Performance Comparison}

We compare {\systemname} with two state-of-the-art sleep monitoring applications. The first is a sleep detection app called ``Sleep As Android", and the second one is a smartphone-based system named Sleep Hunter~\cite{gu2016sleep}. The former app is designed to estimate sleep time and assess sleep by recording the state of motions and the number of body exercises. The latter focuses on estimating sleep stages and evaluate the sleep quality using the tracked sleep-related events. Considering that Sleep As Android can only detect light sleep stage and deep sleep stage, we only compare the performance of these two stages. Table \ref{tab:comparison} shows the detection results. As we can see, {\systemname} significantly outperforms Sleep As Android and deliver better performance than Sleep Hunter. The performance advantage of {\systemname} comes from the incorporation of rich and complicated sleep events.

\begin{table}[!t]\footnotesize
 	\centering
 	\renewcommand\arraystretch{0.3}
 	\caption{Smartwatch based acoustic event detection comparison.}\label{tab:compare_sound1}
 	\begin{tabular}{c| r | r | r | r | r | r | r |}
 		\cline{2-8}
 		&\multicolumn{1}{ c|}{ }
 		&\multicolumn{2}{ c|}{ }
 		&\multicolumn{2}{ c|}{ }
         &\multicolumn{2}{ c|}{ }\\
 		&\multicolumn{1}{c|}{}
 		&\multicolumn{2}{c|}{\textbf{\footnotesize Snore}}
        &\multicolumn{2}{c|}{\textbf{\footnotesize Cough}}
 		&\multicolumn{2}{c|}{\textbf{\footnotesize Somniloquy}} \\
 		\cline{2-8}
 		\multicolumn{1}{c|}{\textbf{}}
 		&  \multicolumn{1}{c|}{\diagbox{System}{Event}}
 		&  \multicolumn{1}{c|}{\footnotesize Precision}
 		&  \multicolumn{1}{c|}{\footnotesize Recall}
        &  \multicolumn{1}{c|}{\footnotesize Precision}
 		&  \multicolumn{1}{c|}{\footnotesize Recall}
 		&  \multicolumn{1}{c|}{\footnotesize Precision}
 		&  \multicolumn{1}{c|}{\footnotesize Recall}\\
 		\cline{2-8}
 		& & & & & & &\\
 		&   \textbf{\footnotesize {\systemname}}   & $96.9\%$    &   $91.4\%$      &   $88.9\%$      &   $90.1\%$  &   $91.3\%$  &   $89.4\%$ \\
 		& & & & & & & \\
 		\cline{2-8}
 		& & & & & & &\\
 		&   \textbf{\footnotesize Sleep Hunter}   &   $71.0\%$      &   $73.0\%$     &   $71.0\%$      &   $63.0\%$   &   $89.0\%$  &   $83.5\%$\\
 		& & & & & & & \\
 		\cline{2-8}
 	\end{tabular}
 \end{table}

\begin{table}[!t]\footnotesize
 	\centering
 	\renewcommand\arraystretch{0.3}
 	\caption{Smartwatch based sleep stage detection comparison.}\label{tab:compare_stage1}
 	\begin{tabular}{c| c | c | c | c | c | c | c |}
 		\cline{2-8}
 		&\multicolumn{1}{ c|}{ }
 		&\multicolumn{2}{ c|}{ }
 		&\multicolumn{2}{ c|}{ }
        &\multicolumn{2}{ c|}{ }\\
 		&\multicolumn{1}{c|}{}
 		&\multicolumn{2}{c|}{\textbf{\footnotesize REM}}
        &\multicolumn{2}{c|}{\textbf{\footnotesize Light Sleep}}
 		&\multicolumn{2}{c|}{\textbf{\footnotesize Deep Sleep}} \\
 		\cline{2-8}
 		\multicolumn{1}{c|}{\textbf{}}
 		&  \multicolumn{1}{c|}{\diagbox{System}{Stage}}
 		&  \multicolumn{1}{c|}{\footnotesize Precision}
 		&  \multicolumn{1}{c|}{\footnotesize Recall}
        &  \multicolumn{1}{c|}{\footnotesize Precision}
 		&  \multicolumn{1}{c|}{\footnotesize Recall}
 		&  \multicolumn{1}{c|}{\footnotesize Precision}
 		&  \multicolumn{1}{c|}{\footnotesize Recall}\\
 		\cline{2-8}
 		& & & & & & &\\
 		&   \textbf{\footnotesize {\systemname}}   & $71.0\%$    &   $70.0\%$      &   $66.5\%$      &   $69.6\%$  &   $63.3\%$  &   $59.8\%$ \\
 		& & & & & & & \\
 		\cline{2-8}
 		& & & & & & &\\
 		&   \textbf{\footnotesize Sleep Hunter}   &   $61.2\%$      &   $67.6\%$     &   $61.5\%$      &   $60.2\%$   &   $56.8\%$  &   $47.9\%$\\
 		& & & & & & & \\
 		\cline{2-8}
 	\end{tabular}
 \end{table}



We also implement the algorithms employed by Sleep Hunter and apply them to the data collected using our smartwatch. This experiment allows us to check if the better performance of {\systemname} is due to the use of a smartwatch instead of a mobile phone.

For body movement detection, applying the algorithms employed by Sleep Hunter to our smartwatch data gives a comparable accuracy of around
96\% when the system only identify between drastic and small body movements. However, Sleep Hunter does not work when we need to detect
finer-grained body movements.  {\systemname} outperforms Sleep Hunter by delivering an accuracy of drastic body movements around 90\% (see
Table~\ref{tab:rollver}) and an accuracy of finer-grained body movements more than 78\% (see Fig.~\ref{fig:micro_combine}). For acoustic
event detections, we apply the Sleep Hunter algorithms to detect snore, cough and somniloquy. The results in Table \ref{tab:compare_sound1}
suggest that {\systemname} gives better performance over Sleep Hunter in detecting these acoustic events. Finally, we apply the sleep stage
detection model used by Sleep Hunter to combine sleep-related events to identify sleep stages. The results are shown in Table
\ref{tab:compare_stage1}. Again,  {\systemname} outperforms the Sleep Hunter model with a higher accuracy and recall.


This experiment confirms that the algorithms used by Sleep Hunter for sleep event and stage detection are not tuned for the smartwatch. Compared to Sleep Hunter, {\systemname} can detect sleep events and stages with a higher accuracy using a set of carefully designed methods target to target smartwatch.


\subsubsection{User Survey}\label{sec:user_survey}

To understand and verify how the additional information captured by {\systemname} supports users, at the end of the experiments the participants are administrated a survey to the participants asking about their experiences with {\systemname} and their personal sleeping patterns. We combine these results with the subjective sleep quality estimates obtained through the PSQI questionnaires administered during the study. We considered two groups of users in our survey. As the main source of information, we consider the $15$ participants in our experiments who were asked about their experiences with {\systemname}, their subjective sleep quality assessment, and details of their personal sleep patterns. This set of users was augmented with external participants who were asked about their interested in the events that {\systemname} is capable of detecting. The questions in our survey include:
\begin{enumerate}
  \item Subjective sleep quality (5-levels, 1 for excellent and 5 for worst),
  \item Sleep duration,
  \item Sleep disturbances,
  \item Daytime dysfunction.
\end{enumerate}
For the above four items, each one is rated on a 1 to 5 scale. These scores are first summed to yield a total score, which ranges from 0 to 20. Then we merge every five neighboring scores into one scale and eventually divide the total scores into four levels, recorded as 0, 1, 2 and 3, representing poor, general, good and excellent, respectively. This step is necessary to compare the results of the user survey against the sleep quality estimation provided by {\systemname} and Fitbit.

\begin{table} \footnotesize
\setlength{\tabcolsep}{1pt}
\renewcommand{\arraystretch}{0.9}{\multirowsetup}{\centering}
\caption{{Results of sleep quality assessments. The first three rows show the sleep quality scores of the different systems (mean and standard deviation) for each user across $14$ days, whereas the last two rows  compare sleep quality labels between subjective assessments and those returned by {\systemname} and FitBit. }}\label{tab:quality}
\begin{tabularx}{\textwidth}{X cccccccccccccccc }
        \toprule
         \textbf{User ID} & 1 & 2 & 3 & 4 & 5 & 6 & 7 & 8 & 9 & 10 & 11 & 12 & 13 & 14 & 15\\
         \midrule
         \rowcolor{Gray}{\systemname} & 3 (1.8) & 3 (1.5) & 0 (1.0) & 1 (1.8) &  2 (1.2) &  2 (1.7) &  3 (1.0) & 0 (1.3) &  2 (1.3) &  2 (1.0) & 2 (0) & 2 (1.5) &  1 (1.8) &  0 (1.0) &  2 (1.3)\\
         Fitbit & 3 (1.7) & 3 (2.2) & 0 (1.3) & 0 (1.8) & 1 (2.2) & 3 (1.8) & 2 (1.9) & 3 (1.8) & 2 (1.7) & 2 (1.9) & 2 (1.3) & 2 (2.0) & 2 (2.3) & 1 (1.8) & 2 (1.7) \\
         \rowcolor{Gray} User score & 3 (0.0) & 2 (1.4) & 0 (0.0) &  0 (1.7) & 2 (1.7) &  2 (1.0) & 3 (1.0) & 1 (1.7) & 1 (1.4) & 2 (1.0) &  2 (1.0) & 3 (2.0) & 0 (1.8) & 0 (1.0) & 2 (1.0)\\
         {\systemname} & P&O&P&O&P&P&P&O&O&P&P&O&O&P&P\\
         \rowcolor{Gray} Fitbit & P&O&P&P&O&O&O&\textbf{B}&O&P&P&O&\textbf{B}&O&P\\
         \bottomrule
\end{tabularx}
\end{table}

In Table \ref{tab:quality}, we show the mean sleep quality score across 14 days given by each user together with the estimations produced by {\systemname} and Fitbit. We also give the standard deviation of the scores across the 14-day period per user. This number is given in the brackets next to the mean score. The last two rows in Table~\ref{tab:quality} compare the estimation given by {\systemname} and Fitbit against the user self-rating score. In these two specific rows, a label of `P' indicates an estimated score perfectly matches the user score, a label of `O' means the estimation error is within one scale point (for example, {\systemname} rates the user sleep to be excellent while the user's self-rating is good), and a label of `B' indicates the estimation error is greater than a scale point. As can be seen from the table, the estimation given by {\systemname} is more likely to match the user's self-rating compared to Fitbit (as indicated by having more `P' labels - 9 vs 6 ) and, unlike Fitbit, the estimation error given by {\systemname} is never greater than one scale point. To further validate this, we calculated the Spearman $\rho$-correlation~\cite{richardson2015nonparametric} between the user scores and each of the two systems, {\systemname} and Fitbit. {\systemname} provides higher correlation coefficient ($\rho = 0.842$) than Fitbit ($\rho = 0.500$). The difference in correlation was found statistically significant using a one-tailed test carried out through a Fisher r-z transformation ($Z = 1.66, p < 0.05$).  In summary, {\systemname} thus gives a better sleep quality assessment compared to Fitbit in our evaluation.

While the results above demonstrate that {\systemname} is capable of accurately estimating sleep quality, the main benefit from {\systemname} compared to previous works is not its sleep quality performance but its capability to analyze and capture the {\em root cause} of sleep issues. To demonstrate this, we carried out a follow-up analysis where we examined the events captured by {\systemname} for each of the six users assessing their sleep quality negatively (poor or general subjective quality, i.e., rating 0 or 1 in Table~\ref{tab:quality}). For four of the six users, we were able to find clear causes for their poor sleep. For one of the users, {\systemname} indicated difficulties in falling asleep, which was reflected in a high body rollover count. Further analysis of captured events indicated ambient noise and lighting to be most likely reasons for this participant. Another user complained of feeling of numbness in the arm after sleep. Events captured by {\systemname} showed that this was likely due to bad hand posture as the person tended to put the hand on top of the head before sleeping. The third user complained of frequent nightmares. Analysis of {\systemname} events showed that the person habitually slept on the left side, which has been shown to have the higher risk on nightmares~\cite{nightmare}, and often placed a hand on top of the chest, which creates additional pressure and can lead to nightmares. Finally, one of the users mentioned suffering from long-term snoring problems, which we also were able to detect from the events captured by {\systemname}. The events also highlighted that the person was often sleeping in the supine position, which further increases susceptibility to snoring-related problems. Existing systems are only capable of capturing some of these factors influencing sleep quality and thus they are not capable of providing a holistic view of the participant's sleep quality, whereas {\systemname} is capable of providing very detailed information about sleep events. To further demonstrate the benefits of {\systemname} compared to previous works, we asked the 15 participants to make appropriate adjustments according to our recommendations and to conduct a return visit survey three weeks later. It was found that some of the users were able to reduce symptoms and to improve their average quality of sleep based on the suggestions.
	
As for the user experience, results from the survey highlighted a strong interest in the information captured by {\systemname}. In particular, 80\% of participants believe that the detection of sleep posture is very necessary, showing their sleep posture can not only help people to avoid health problems caused by long-term improper sleeping posture, but also help us find out the reasons for the next day's physical discomfort, such as dizziness, muscle soreness may be due to improper sleeping posture. And there are some users are troubled by snoring. This may be due to improper sleeping posture. We map the detected snoring event and sleeping posture to suggest the user to modify his posture to a suitable posture to reduce the harm caused by long-term snoring. 60\% of the participants thought it useful to detect the hand position in supine posture, even one user mentioned that he did often have nightmares and our system found his hand was often placed on his chest, and then {\systemname} could remind him that he should take some measures to avoid such a position and thus reduce the poor sleep quality that nightmare brings. Only 20\% of participants found it useful to calculate the number of body rollover. However, detection of rollovers is useful in segmenting sleep stages. Furthermore, body rollover counts could be used to derive additional information to the user, such as how restless or peaceful the sleep has been overall.
