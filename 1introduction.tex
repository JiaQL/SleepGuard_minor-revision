\section{INTRODUCTION}\label{sec:1introduction}

Sleep plays a vital role in good health and personal well-being throughout one's life. Lack of sleep or poor quality of sleep can lead to
serious, sometimes life-threatening, health problems~\cite{altena2008sleep,chandola2010effect,lallukka2016contribution}, decrease level of
cognitive performance~\cite{alhola07sleep,akerstedt07altered}, and affect mood and feelings of personal
well-being~\cite{paunio09longitudinal,pilcher97sleep}. Besides having an adverse effect on individuals, insufficient or poor quality sleep
has a significant economic burden, among others, through decreased productivity, and medical and social costs associated with treatment of
sleep disorders~\cite{hafner17why}. Indeed, to highlight the significance of sleep quality, the Centre for Disease Prevention (CDC) has
declared insufficient sleep as a public health problem in the US~\cite{sleepreport}, and the concern is widely shared amongst other
 countries.


Traditionally, sleep monitoring is performed in a clinical environment using Polysomnography (PSG). In PSG, medical
sensors attached to human body are used to monitor events and information such as respiration, electroencephalogram (EEG), electrocardiogram (ECG), electro-oculogram and oxygen saturation~\cite{ebrahimi2008automatic,saper2005hypothalamic,oropesa1999sleep,langkvist2012sleep}. These information sources can then be used to determine sleep stages, sleep efficiency, abnormal breathing, and overall sleep quality. PSG is widely considered as the gold standard for sleep monitoring, and while it is extensively used to support clinical treatments of sleep disorders, it has some disadvantages that make it unsuitable for longitudinal and large-scale sleep monitoring. Firstly, {attaching and outfitting} the sensing instruments is time-consuming and laborious, and they are prone to disrupting sleeping routines. Secondly, PSG is rather expensive to use and requires a clinical environment and highly trained medical professionals to operate. Due to these disadvantages, PSG is only suitable as a way to support severe disorders {where} clinical care is required.

Recently, sleep monitoring based on off-the-shelf mobile and wearable devices has emerged as an alternative way to obtain information about one's sleeping patterns~\cite{ko15consumer,shelgikar2016sleep}. By taking advantage of diverse sensors, behaviors and routines associated with sleeping can be captured and modelled. This in turn can help users understand their sleep behavior and provide feedback on how to improve sleep, for example, by changing routines surrounding sleep activity or improving the sleeping environment. What makes self monitoring particularly attractive is the non-invasive nature of the sensing compared to PSG. Examples of consumer-grade sleep monitors range from apps running on smartphones or tablets to smartwatches and specialized wearable devices~\cite{zeo,Jawbone,SleepAndroid,fitbit,gu2016sleep,sleepmonitor}.

Despite the popularity of consumer-grade sleep monitors, currently the full potential of these devices is not being realized. Indeed, while current consumer-grade sleep monitors can capture and model a wide range of sleep related information, such as estimating overall sleep quality, capturing different stages of sleep, and identifying specific events occurring during
sleep~\cite{kay2012lullaby,zhang2013real,sleepmonitor}, they offer little help in understanding the characteristics that surround poor sleep. Thus, these solutions are unable to capture the root cause behind poor sleep or to provide {actionable} recommendations on how to improve sleep quality. This is because current solutions focus on monitoring characteristics of the sleep itself, without considering behaviors occurring during sleep and the environmental context affecting sleep, e.g., ambient light-level and noise. Indeed, sleep quality has been shown to depend on a wide range of factors. For example, intensity of ambient light~\cite{hood04determinants} and noisiness~\cite{muzet2007environmental} of the environment can significantly affect sleep quality. Similarly, the user's breathing patterns, postures during sleep, and routines surrounding the bedtime also have a significant impact on sleep quality. Without details of the environment and activities across sleep stages, the root cause of poor sleep cannot be captured and the user informed of how to improve their sleep quality. To unlock the full potential of consumer-grade sleep monitoring, innovative ways to take advantage of the rich sensor data accessible through these devices are required.

This paper presents the design and development of {\systemname}, a \emph{holistic sleep monitoring solution} that captures rich information
about sleep events, the sleep environment, and the overall quality of sleep. {\systemname} is the first to solely rely on sensor
information available on off-the-shelf smartwatches for capturing a wide range of sleep-related activities (see Table~\ref{tab:test}). The
key insight in {\systemname} is that sleep quality is strongly correlated with characteristics of body movements, health related factors
that can be identified from audio information, and characteristics of the sleep environment~\cite{shelgikar2016sleep}. By using a
smartwatch, the sensors are close to the user during all stages during the sleep, enabling detailed capture of not only sleep cycles, but
body movements and environmental changes taking place during the sleep period. Capturing these sleeping events from sensor data, however,
is non-trivial due to changes in sensor measurements caused by hand motions during sleep. To overcome this challenge, changes in sensor
orientation relative to the user's body need to be tracked and opportune moments where to capture sensor data need to be detected.
{\systemname} addresses these issues by integrating a set of new methods for analyzing and capturing sleep-related information from sensor
measurements available on a smartwatch. {\systemname} also incorporates a model that uses the detected events to infer the user's sleep
stages and sleep quality.

While some prior research has examined the use of smartwatches for sleep
monitoring~\cite{pombo2016ubisleep,shelgikar2016sleep,haescher2015anomaly,borazio2012combining}, these approaches have only been able to
gather coarse-grained information about sleep and often required additional highly-specialized devices, such as pressure mattresses or
image acquisition equipment to supplement the measurements available from the smartwatch. In this paper, we demonstrate that, for the first
time, using {\em only a smartwatch}, it is possible to capture an extensive set of sleep-related information -- many of which are not
presented in prior work. Having a more comprehensive set of sleep-related events and activities available enables users to gain a deeper
understanding of their sleep patterns and the causes of poor sleep, and to make recommendations on how to improve one's sleep quality.

We evaluate {\systemname} through rigorous and extensive benchmark experiments conducted on data collected from 15 participants during a
two week monitoring period. The results of our experiments demonstrate that {\systemname} can accurately characterize body motions and
movements during sleep, as well as capture different acoustic events. Specifically, the lowest event-detection accuracy for {\systemname}
in our experiments is 87\%, with the best event detection accuracy reaching up to 98\%. We also demonstrate that {\systemname} can
accurately detect various sleep stages and help users to better understand their sleep quality. During our experiments, $6$ of the $15$
participants suffered from some sleep problems ($4$ with bad and $2$ with general sleep quality), all of whom were correctly identified by
{\systemname}. Moreover, we also demonstrate that {\systemname} is able to correctly identify the root causes of sleep problems for the $4$
participants with bad sleep quality, whether it is due to suboptimal hand position, body posture or sleeping environment. Compared to
state-of-the-art sleep monitoring systems, such as Fitbit and Sleep Hunter, the main advantage of {\systemname} is that can report a wider
range of sleep events and provide a better understanding for the causes of sleep problems.

\begin{table}[!t]
 \caption{\label{tab:test}Sleep events targeted in this work}
 \centering
 \small
 \begin{tabular}{ll}
  %\toprule
  \toprule
  \textbf{Event}& \textbf{Type} \\
  \midrule
\rowcolor{Gray}  Sleep postures & Supine, Left lateral, Right lateral, Prone\\
 Hand positions & Head, Chest, Abdomen\\
\rowcolor{Gray} Body rollover & Count\\
 Micro body movements& Hand moving, Arm raising, Body trembling \\
\rowcolor{Gray} Acoustic events & Snore, Cough, Somniloquy  \\
 Illumination condition & Strong, Weak  \\
  \bottomrule
 %\hline
 \end{tabular}
\end{table}


%\subsection*{Contributions}
This paper makes the following contributions:

\begin{itemize}
	\item We present the design and development of {\systemname}, the first holistic sleep monitoring system to rely solely on sensors
available in an off-the-shelf smartwatch to capture a wide range of sleep information that characterizes overall sleep quality, user
behaviors during sleep, and the sleep environment.
	
\item We develop novel and lightweight algorithms for capturing sleep-related information on smartwatches taking into consideration changes
in orientation and location of the device during different parts of the night. We show how to overcome specific challenges to effectively
track events like sleep postures (Sec.~\ref{sec:sleeppdet}), hand positions (Sec.~\ref{sec:handpr}), body rollovers
(Sec.~\ref{sec:bodyrollover}), micro body movements (Sec.~\ref{sec:microbo}), and acoustic events (Sec.~\ref{sec:acoustic}) and
illumination conditions (Sec.~\ref{sec:illumination}).
	

    \item We extensively evaluate the performance of {\systemname} using measurements collected from two-week monitoring of $15$
    participants (Sec.~\ref{sec:expsetup}). Our results demonstrate that {\systemname} can accurately capture a wide range of sleep events,
    estimate different sleep stages, and produce meaningful information about overall sleep quality (Sec.~\ref{sec:4experiment}).  We show
    that {\systemname} successfully reveals the causes of poor sleeps for some of our testing users and subsequently helps them improve
    their sleep by changing their sleep behaviors and sleeping environment  (Sec.~\ref{sec:user_survey}).
\end{itemize}
