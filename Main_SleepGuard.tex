\documentclass[acmlarge]{acmart}

\usepackage{booktabs} % For formal tables
\usepackage[ruled]{algorithm2e} % For algorithms
\usepackage{colortbl}
\usepackage{subfigure}
\usepackage{multirow}
\usepackage{enumitem}
\usepackage{rotating}
%\usepackage{times}  %time new roman type
\usepackage{url}
\usepackage{subfigure}
\usepackage{multirow}
\usepackage{tabularx}
\usepackage{diagbox}
\usepackage{color}

\newcommand{\systemname}{SleepGuard}
\definecolor{Gray}{gray}{0.9}
\newcommand{\etal}{\emph{et al.}}

\SetAlFnt{\small}
\SetAlCapFnt{\small}
\SetAlCapNameFnt{\small}
\SetAlCapHSkip{0pt}
\IncMargin{-\parindent}

% Metadata Information
%%% Revise this part at the final submission
\acmJournal{IMWUT}
\acmVolume{2}
\acmNumber{3}
\acmArticle{39}
\acmYear{2018}
\acmMonth{9}
\acmArticleSeq{11}
% Copyright
\setcopyright{acmcopyright}
%\setcopyright{acmlicensed}
%\setcopyright{rightsretained}
%\setcopyright{usgov}
\setcopyright{usgovmixed}
%\setcopyright{cagov}
%\setcopyright{cagovmixed}
\acmPrice{15.00}

%%% Revise this part at the final submission
\acmDOI{0000001.0000001}

% Paper history
%% Uncomment and revise this part at the final submission
\received{February 2018}
\received{May 2018}
\received[accepted]{August 2018}



\begin{document}

\title{\systemname: Capturing Rich Sleep Information Using Smartwatch Sensing Data}
%\titlenote{We can add a note to the title}

%% Uncomment  this part at the final submission
\author{Liqiong Chang, Jiaqi Lu, Ju Wang}
\author{Xiaojiang Chen}\authornote{This is the corresponding author}
\author{Dingyi Fang, Zhanyong Tang}
\orcid{1234-5678-9012-3456}
\affiliation{
	\institution{Northwest University}
	\country{China}} \email{{clq, jqlu, wangju, xjchen, dyf, zytang}@nwu.edu.cn}

\author{Petteri Nurmi$^{\dag\ddag}$}
\affiliation{
	\institution{Lancaster University$^{\dag}$}
	\country{United Kingdom},
  \institution{University of Helsinki$^{\ddag}$}
	\country{Finland}} \email{{p.nurmi}@lancaster.ac.uk}

\author{Zheng Wang}
\affiliation{
	\institution{University of Helsinki}
	\country{United Kingdom}} \email{{z.wang}@lancaster.ac.uk}

%NOTE: This is supposed to be anonymous review.

\begin{abstract}
Sleep is an important part of our daily routine -- we spend about one-third of our time doing it. By tracking sleep-related events and activities, sleep monitoring provides decision support to help us understand sleep quality and  causes of poor sleep. Wearable devices provide a new way for sleep monitoring, allowing us to monitor sleep from the comfort of our own home. However, existing  solutions do not take full advantage of the rich sensor data provided by these devices. In this paper, we present the design and development of {\systemname}, a novel approach to track a wide range of sleep-related events using smartwatches. We show that using merely a single smartwatch, it is possible to capture rich amount of information about sleep events and sleeping context, including body posture and movements, acoustic events, and illumination conditions. We demonstrate that through these events it is possible to estimate sleep quality and identify factors affecting it most. We evaluate our approach by conducting extensive experiments involved fifteen users across a 2-week period. Our experimental results show that our approach can track a richer set of sleep events, provide better decision support for evaluating sleep quality, and help to identify causes for sleep problems compared to prior work. We also show that {\systemname} can help users to improve their sleep quality by helping them to understand root causes of sleep problems.
\end{abstract}


\begin{CCSXML}
	<ccs2012>
	<concept>
	<concept_id>10003120.10003138</concept_id>
	<concept_desc>Human-centered computing~Ubiquitous and mobile computing</concept_desc>
	<concept_significance>500</concept_significance>
	</concept>
	</ccs2012>
\end{CCSXML}

\ccsdesc[500]{Human-centered computing~Ubiquitous and mobile computing}

\keywords{Smartwatch, Sleep Events, Sensing}

\thanks{This work is partially supported by the xxx.}

\maketitle

\renewcommand{\shortauthors}{L. Chang et al. }


\section{INTRODUCTION}\label{sec:1introduction}

Sleep plays a vital role in good health and personal well-being throughout one's life. Lack of sleep or poor quality of sleep can lead to
serious, sometimes life-threatening, health problems~\cite{altena2008sleep,chandola2010effect,lallukka2016contribution}, decrease level of
cognitive performance~\cite{alhola07sleep,akerstedt07altered}, and affect mood and feelings of personal
well-being~\cite{paunio09longitudinal,pilcher97sleep}. Besides having an adverse effect on individuals, insufficient or poor quality sleep
has a significant economic burden, among others, through decreased productivity, and medical and social costs associated with treatment of
sleep disorders~\cite{hafner17why}. Indeed, to highlight the significance of sleep quality, the Centre for Disease Prevention (CDC) has
declared insufficient sleep as a public health problem in the US~\cite{sleepreport}, and the concern is widely shared amongst other
 countries.


Traditionally, sleep monitoring is performed in a clinical environment using Polysomnography (PSG). In PSG, medical
sensors attached to human body are used to monitor events and information such as respiration, electroencephalogram (EEG), electrocardiogram (ECG), electro-oculogram and oxygen saturation~\cite{ebrahimi2008automatic,saper2005hypothalamic,oropesa1999sleep,langkvist2012sleep}. These information sources can then be used to determine sleep stages, sleep efficiency, abnormal breathing, and overall sleep quality. PSG is widely considered as the gold standard for sleep monitoring, and while it is extensively used to support clinical treatments of sleep disorders, it has some disadvantages that make it unsuitable for longitudinal and large-scale sleep monitoring. Firstly, {attaching and outfitting} the sensing instruments is time-consuming and laborious, and they are prone to disrupting sleeping routines. Secondly, PSG is rather expensive to use and requires a clinical environment and highly trained medical professionals to operate. Due to these disadvantages, PSG is only suitable as a way to support severe disorders {where} clinical care is required.

Recently, sleep monitoring based on off-the-shelf mobile and wearable devices has emerged as an alternative way to obtain information about one's sleeping patterns~\cite{ko15consumer,shelgikar2016sleep}. By taking advantage of diverse sensors, behaviors and routines associated with sleeping can be captured and modelled. This in turn can help users understand their sleep behavior and provide feedback on how to improve sleep, for example, by changing routines surrounding sleep activity or improving the sleeping environment. What makes self monitoring particularly attractive is the non-invasive nature of the sensing compared to PSG. Examples of consumer-grade sleep monitors range from apps running on smartphones or tablets to smartwatches and specialized wearable devices~\cite{zeo,Jawbone,SleepAndroid,fitbit,gu2016sleep,sleepmonitor}.

Despite the popularity of consumer-grade sleep monitors, currently the full potential of these devices is not being realized. Indeed, while current consumer-grade sleep monitors can capture and model a wide range of sleep related information, such as estimating overall sleep quality, capturing different stages of sleep, and identifying specific events occurring during
sleep~\cite{kay2012lullaby,zhang2013real,sleepmonitor}, they offer little help in understanding the characteristics that surround poor sleep. Thus, these solutions are unable to capture the root cause behind poor sleep or to provide {actionable} recommendations on how to improve sleep quality. This is because current solutions focus on monitoring characteristics of the sleep itself, without considering behaviors occurring during sleep and the environmental context affecting sleep, e.g., ambient light-level and noise. Indeed, sleep quality has been shown to depend on a wide range of factors. For example, intensity of ambient light~\cite{hood04determinants} and noisiness~\cite{muzet2007environmental} of the environment can significantly affect sleep quality. Similarly, the user's breathing patterns, postures during sleep, and routines surrounding the bedtime also have a significant impact on sleep quality. Without details of the environment and activities across sleep stages, the root cause of poor sleep cannot be captured and the user informed of how to improve their sleep quality. To unlock the full potential of consumer-grade sleep monitoring, innovative ways to take advantage of the rich sensor data accessible through these devices are required.

This paper presents the design and development of {\systemname}, a \emph{holistic sleep monitoring solution} that captures rich information
about sleep events, the sleep environment, and the overall quality of sleep. {\systemname} is the first to solely rely on sensor
information available on off-the-shelf smartwatches for capturing a wide range of sleep-related activities (see Table~\ref{tab:test}). The
key insight in {\systemname} is that sleep quality is strongly correlated with characteristics of body movements, health related factors
that can be identified from audio information, and characteristics of the sleep environment~\cite{shelgikar2016sleep}. By using a
smartwatch, the sensors are close to the user during all stages during the sleep, enabling detailed capture of not only sleep cycles, but
body movements and environmental changes taking place during the sleep period. Capturing these sleeping events from sensor data, however,
is non-trivial due to changes in sensor measurements caused by hand motions during sleep. To overcome this challenge, changes in sensor
orientation relative to the user's body need to be tracked and opportune moments where to capture sensor data need to be detected.
{\systemname} addresses these issues by integrating a set of new methods for analyzing and capturing sleep-related information from sensor
measurements available on a smartwatch. {\systemname} also incorporates a model that uses the detected events to infer the user's sleep
stages and sleep quality.

While some prior research has examined the use of smartwatches for sleep
monitoring~\cite{pombo2016ubisleep,shelgikar2016sleep,haescher2015anomaly,borazio2012combining}, these approaches have only been able to
gather coarse-grained information about sleep and often required additional highly-specialized devices, such as pressure mattresses or
image acquisition equipment to supplement the measurements available from the smartwatch. In this paper, we demonstrate that, for the first
time, using {\em only a smartwatch}, it is possible to capture an extensive set of sleep-related information -- many of which are not
presented in prior work. Having a more comprehensive set of sleep-related events and activities available enables users to gain a deeper
understanding of their sleep patterns and the causes of poor sleep, and to make recommendations on how to improve one's sleep quality.

We evaluate {\systemname} through rigorous and extensive benchmark experiments conducted on data collected from 15 participants during a
two week monitoring period. The results of our experiments demonstrate that {\systemname} can accurately characterize body motions and
movements during sleep, as well as capture different acoustic events. Specifically, the lowest event-detection accuracy for {\systemname}
in our experiments is 87\%, with the best event detection accuracy reaching up to 98\%. We also demonstrate that {\systemname} can
accurately detect various sleep stages and help users to better understand their sleep quality. During our experiments, $6$ of the $15$
participants suffered from some sleep problems ($4$ with bad and $2$ with general sleep quality), all of whom were correctly identified by
{\systemname}. Moreover, we also demonstrate that {\systemname} is able to correctly identify the root causes of sleep problems for the $4$
participants with bad sleep quality, whether it is due to suboptimal hand position, body posture or sleeping environment. Compared to
state-of-the-art sleep monitoring systems, such as Fitbit and Sleep Hunter, the main advantage of {\systemname} is that can report a wider
range of sleep events and provide a better understanding for the causes of sleep problems.

\begin{table}[!t]
 \caption{\label{tab:test}Sleep events targeted in this work}
 \centering
 \small
 \begin{tabular}{ll}
  %\toprule
  \toprule
  \textbf{Event}& \textbf{Type} \\
  \midrule
\rowcolor{Gray}  Sleep postures & Supine, Left lateral, Right lateral, Prone\\
 Hand positions & Head, Chest, Abdomen\\
\rowcolor{Gray} Body rollover & Count\\
 Micro body movements& Hand moving, Arm raising, Body trembling \\
\rowcolor{Gray} Acoustic events & Snore, Cough, Somniloquy  \\
 Illumination condition & Strong, Weak  \\
  \bottomrule
 %\hline
 \end{tabular}
\end{table}


%\subsection*{Contributions}
This paper makes the following contributions:

\begin{itemize}
	\item We present the design and development of {\systemname}, the first holistic sleep monitoring system to rely solely on sensors
available in an off-the-shelf smartwatch to capture a wide range of sleep information that characterizes overall sleep quality, user
behaviors during sleep, and the sleep environment.
	
\item We develop novel and lightweight algorithms for capturing sleep-related information on smartwatches taking into consideration changes
in orientation and location of the device during different parts of the night. We show how to overcome specific challenges to effectively
track events like sleep postures (Sec.~\ref{sec:sleeppdet}), hand positions (Sec.~\ref{sec:handpr}), body rollovers
(Sec.~\ref{sec:bodyrollover}), micro body movements (Sec.~\ref{sec:microbo}), and acoustic events (Sec.~\ref{sec:acoustic}) and
illumination conditions (Sec.~\ref{sec:illumination}).
	

    \item We extensively evaluate the performance of {\systemname} using measurements collected from two-week monitoring of $15$
    participants (Sec.~\ref{sec:expsetup}). Our results demonstrate that {\systemname} can accurately capture a wide range of sleep events,
    estimate different sleep stages, and produce meaningful information about overall sleep quality (Sec.~\ref{sec:4experiment}).  We show
    that {\systemname} successfully reveals the causes of poor sleeps for some of our testing users and subsequently helps them improve
    their sleep by changing their sleep behaviors and sleeping environment  (Sec.~\ref{sec:user_survey}).
\end{itemize}

\section{{\systemname} SLEEP MONITORING PLATFORM}\label{Sec:2design}

{\systemname} is a novel smartwatch-based sleep monitoring system that aims at estimating sleep quality and capturing rich information about behaviors and events occurring during sleep. By capturing this information, {\systemname} can analyze potential reasons for sleep problems and provide the user with suggestions on how to improve their sleep routine or sleep environment. To achieve its design goals, {\systemname} exploits a wide range of sensors that are common on commercial off-the-shelf smartwatches: (i) accelerometer, gyroscope, and orientation sensor are used to collect body and hand movements; (ii) microphone is used to measure the level of ambient noise and to capture acoustic events; and (iii) ambient light sensor is used to monitor illumination within the sleep environment. The different sensors and information extracted from them are summarized in Table~\ref{tab:test}. In the following, we discuss the different subcomponents of {\systemname} in detail.


\subsection{Detecting Sleep Postures and Movements}

One's sleeping position, referred to as {\em sleep posture}, and the extent of body movements are important factors in determining overall sleep quality. Suboptimal posture has been shown to affect the severity of sleep disorders and is widely used in medical diagnoses to analyze effects of sleep disorders~\cite{oksenberg1998effect,eiseman2012impact} while having a good sleep posture has been shown to correlate with subjective assessments of sleep quality~\cite{dekoninck83sleep}. Similarly, the high degree of body movements during sleep likely reflects restlessness, which results in poor sleep quality. {\systemname} uses motion sensors (accelerometer, gyroscope, and orientation sensor) to capture the user's sleep posture and habits. In the following, we detail the techniques we use for capturing the body posture and movements.  {\systemname}, currently supports the 4 basic sleep postures (see Fig.~\ref{fig:BodyPosture}); 3 hand positions (see Fig.~\ref{fig:HandPosition}); 6 types of body rollovers (see Fig.~\ref{fig:BodyRollover} for an example); and 3 types of body micro movements.

\begin{figure*}[!t]
	\centering
	\begin{minipage}[t]{.33\textwidth}
		\centering
		  \includegraphics[width=4.7cm,height=3.7cm]{Figures/BodyPosture.pdf}
		\caption{Four sleep body postures.}
		\label{fig:BodyPosture}
	\end{minipage}%
	\begin{minipage}[t]{.33\textwidth}
		\centering
		\includegraphics[width=4.1cm,height=3.4cm]{Figures/HandPosition.pdf}
		\caption{Three hand positions.}
		\label{fig:HandPosition}		
	\end{minipage}
\begin{minipage}[t]{.33\textwidth}
		\centering
	\includegraphics[width=4.7cm,height=3.7cm]{Figures/BodyRollover.pdf}
	\caption{Body rollover from the left side to the right side.}
	\label{fig:BodyRollover}
\end{minipage}
\end{figure*}



\subsubsection{Sleep Posture Detection}\label{sec:sleeppdet}

Dreaming and sleep quality are associated with underlying brain functions, which in turn are affected by body posture~\cite{posture2004}. Sleep posture also varies across individuals and should fit personal and physical needs of the individual~\cite{posture2016,posture2017}. For example, sleeping in a prone position is unsuitable for people with ailments, such as heart disease or high blood pressure. On the other hand, people can {unconsciously} avoid postures that would be beneficial for health and sleep quality~\cite{posture2015}. Having an effective way to detect current posture and track changes in it would thus be essential for estimating overall sleep quality, and avoiding potential harm. {{\systemname} captures four basic sleep postures: supine, left lateral, right lateral, and prone. These are illustrated in} Fig.~\ref{fig:BodyPosture}. Detecting these postures using a single wrist sensor, however, is non-trivial because the sensor cannot accurately track movement of the entire body. To accomplish posture detection, we observe that the arm position strongly correlates {with} sleep posture, i.e., the arm is typically located in a specific, stable location for a given posture. This  suggests that we can first identify the user's arm position and the {\em time the position is approximately stable, which can then be mapped into} a sleep posture. Later in this paper, we show that our approach {achieves} high accuracy in identifying sleep postures.

To separate sleep postures, {\systemname} considers a set of feasible hand positions for each posture. {In the supine position, we assume the user's hand to be on the left side of the body, on the abdomen, on the chest or on the head; in the left and right lateral positions we assume the hand to be close to the pillow, on the chest or on the waist; and, finally, in the prone position we assume the user's hand is on the side of the head or above his/her head.} These positions were selected based on a pilot carried out in our test environment (see Sec.~\ref{sec:expsetup}). Fig.~\ref{fig:BodyPosture} shows one possible hand position for each of the postures.

Like to SleepMonitor~\cite{sleepmonitor}, we use {three dimensional tilt angles} to detect postures. To identify {which posture the data collected within a time window corresponds to}, we average all the calculated tilt angle values of that window in each dimension. We then calculate the Euclidean distance of the input values to a set of posture profiles, which are based on measurements collected in a pilot study that involved 10 users (see Sec.~\ref{sec:trainingdata}). We then use the body posture associated with the nearest neighbor as the detection outcome. Fig.~\ref{fig:posture} shows the angle values of the four sleep postures targeted in this work. {We can observe clear differences in the tilt angles of the three axes}. The sleep posture thus can be inferred based on the position of the smartwatch and the created angle mapping. However, a limitation of this approach is that the hand positions during supine and prone postures {can be} similar when the hand is located on the side of the head (Fig. \ref{fig:Supine} and \ref{fig:Prone}), thus the classification accuracy will be affected.

\begin{figure*}[!t]
	\centering
	\subfigure[Supine]{
		\label{fig:Supine}
		\includegraphics[width=0.237\columnwidth]{Figures/Supine.pdf}}
	\subfigure[Left Lateral]{
		\label{fig:LeftLateral}
		\includegraphics[width=0.237\columnwidth]{Figures/LeftLateral.pdf}}
	\subfigure[Right Lateral]{
		\label{fig:RightLateral}
		\includegraphics[width=0.237\columnwidth]{Figures/RightLateral.pdf}}
	\subfigure[Prone]{
		\label{fig:Prone}
		\includegraphics[width=0.237\columnwidth]{Figures/Prone.pdf}}
	\caption{The tilt angle characteristics of four body postures.}\label{fig:posture}
\end{figure*}


To improve detection accuracy between supine and prone postures, {\systemname} integrates orientation data as an auxiliary feature. This is {motivated by the observation that hand directions differ in supine and prone positions}. When the result of the previous step is prone or supine and hand is detected to be located next to the body, we combine the tilt angle with three axes data obtained from the direction sensor as a new feature, and classify these postures using a template-based distance matching approach. Specifically, we return the position corresponding to the template with minimum Euclidean distance with current sensor measurements as the user's posture. When we use the direction sensor, we must limit the pillow orientation remaining unchanged (in the experiment our pillow is placed on the north). In fact, this assumption can be easily satisfied since most people usually have fixed sleep directions.


\subsubsection{Hand Position Recognition\label{sec:handpr}}

The hand position during sleep can disclose potential health problems, and an improper hand position can even result in health issues~\cite{position2014}. For instance, placing the hand on the abdomen may indicate discomfort whereas placing the hand on the chest can increase the likelihood of nightmares due to long-term pressure on the heart. Similarly, placing the hand on the head can put excess pressure on shoulder nerves and cause arm pain as blood flow is restricted. This can lead to eventual nerve damage, with symptoms including a tingling sensation and numbness \cite{position2014}.


{\systemname} is designed to recognize three common hand positions -- if the hand is placed on the abdomen, chest or head when the user is in the supine posture, as shown in Fig. \ref{fig:HandPosition}. We have chosen these three hand positions because there are found to be the most common and representative positions in our pilot study (Sec.~\ref{sec:trainingdata}). Our hand position recognition algorithm is based on sensor data of rotation angles, tilt angles, and respiratory events. It works by first using the rotation and tile angles to detect if the hand was placed on the head. {When the hand is not on the head, we use respiratory events to identify whether the hand is on the abdomen or chest}. We now describe how to detect each of the three positions in more details.


\cparagraph{Detect the hand position.}  Fig.~\ref{Bodyhand} shows change of rotation angle using the gyroscope for one of our pilot study
users when his hand was initially placed next to the body and then moved to his head, abdomen, and chest. As can be seen from the figure,
when the hand is moved to the head, changes in rotation angles are significantly different from readings {compared to moving the hand on
abdomen or chest}. This is largely due to the {palm facing upward when the hand is placed on the head compared to it facing downward in
other positions}. {\systemname} exploits this observation to detect if the hand is placed on the head by examining changes of the {tilt}
and rotation angles. We use a hierarchical classifier consisting of two {k nearest neighbor} models (with $k=1$) to predict if the hand is
moved to the head based on the tilt and rotation angle readings. Specifically, we use the first kNN model to detect if the input tilt angle
reading is closest to one of our training samples where the palm was {facing up. Training data for detecting palm direction (upward or
downward) are collected} from our pilot study users when their hands are placed on the head, the abdomen and the chest respectively; see
Sec.~\ref{sec:trainingdata}) If the first kNN model suggests that the palm was facing upward, {the second kNN uses differences of rotation
readings (from the x, y, and z directions) before and after hand  movement to determine if the most likely position is on the head or
elsewhere. As similarity measure we consider}  Euclidean distance from the input data to each of the training samples -- consisting of the
rotation angle values from the three directions.

If our hierarchical model predicts that the hand was not placed on the head, we then use the method
described in the next paragraph to detect if it was placed on the abdomen or the chest.

\begin{figure*}[!t]
	\centering
	\subfigure[moving to the head]{\label{BodytoHead}
		\includegraphics[width=0.32\linewidth]{Figures/BodytoHead.pdf}}
	\subfigure[moving to the abdomen]{\label{BodytoAbdomen}
		\includegraphics[width=0.32\linewidth]{Figures/BodytoAbdomen.pdf}}
	%  \hfill
	\subfigure[moving to the chest]{\label{BodytoChest}
		\includegraphics[width=0.32\linewidth]{Figures/BodytoChest.pdf}}
	\caption{The differences of the rotation angle when a hand of one of our users, placed next to the subject's body, is moved to his head (a), abdomen (b), and chest (c). }\label{Bodyhand}
\end{figure*}



\cparagraph{Detect abdomen and chest positions.} {When the hierarchical model predicts the hand to placed elsewhere than the head, it can
be located at anywhere, including different parts of the body. {However, in our case we are only interested in detection if the hand is on
the chest or abdomen -- or at neither of these positions.} We build on the intuition that the hand is likely to be affected by breathing
whenever it is placed on the chest or abdomen.} Specifically, the impact of breathing results in periodical fluctuations on the
accelerometer readings. This is because the hand will be pushed up and drop down due to breathing.  {Our experimental data suggest that
this behavior only takes place when the hand is placed on the abdomen or the chest, not when the hand is located elsewhere (such as the
shoulder). Thus we can separate these two locations from other locations by examining whether the accelerometer values are impacted by
respiration.}

To examine if the accelerometer data are affected by the respiration, we calculate the power spectral density (PSD) of the collected accelerometer data.  {We then check the PSD to see if we can observe any peak that closely matches human respiratory frequency}. A match indicates that the hand is affected by a respiratory event and hence the hand is likely to be placed on either the abdomen or the chest. Fig. \ref{fig:PSD} provides an empirical evidence to support our design choice. It shows the PSD for one of pilot study user when his hand was placed on the chest. Here we calculate the PSD for the accelerometer data collected from the x, y and z directions. We can see from the diagram that there is a large peak located at around 0.25Hz (highlighted in the diagrams) when a respiratory event is detected (which was verified by video feed). This peak corresponds to the average respiratory frequency of an adult (0.2Hz to 0.47Hz)~\cite{Breath_frequence}, suggesting that the PSD reading can be used as a proxy to detect respiratory events. {\systemname} thus exploits the PSD to detect if the hand is placed on the abdomen or the chest by checking if there is any peak value of PSD falls within the range of 0.2Hz (corresponding to 12 breathes per minute) and 0.47Hz (corresponding to 28 breathes per minute).

\begin{figure*}[!t]
	\centering
	\begin{minipage}[t]{.4\textwidth}
	\centering
\includegraphics[width=7cm,height=5.7cm]{Figures/PSD.pdf}
\caption{The power spectral density (PSD) of the accelerometer readings when a user's hand is placed on his chest.}\label{fig:PSD}
	\end{minipage}%
\hfill
	\begin{minipage}[t]{.55\textwidth}
	\centering
	\includegraphics[width=8cm,height=5.7cm]{Figures/breath_ok1.pdf}
	\caption{The periodic change of the acceleration signal. (a) REM--Location 1. (b) REM--Location 2.  (c) NREM--Location 1.}\label{fig:breath_ok1}
	\end{minipage}
\end{figure*}

{Putting things together, we thus first use the PSD detector to identify whether a respiratory event is taking place. When respiratory events are detected, we }then use again a kNN classifier to make a binary decision to determine if the hand is placed on the abdomen or the chest based on the rotation angle readings (see Fig.~\ref{Bodyhand} b and c). {On the other hand, if no respiratory peak is detected, we assume the hand to be located at another place on the body that is not supported. } Similarly to the head position model, the training samples for the kNN model are also collected from our pilot study users -- where each training example includes the rotation angle readings when the hand is either placed on the abdomen or the chest.

\subsubsection{Labeling REM/non-REM Stages}

We have also found that the extent of body movements can be used to judge the amplitude of respiration, which in turns allows us to detect rapid-eye-movement (REM) and non-rapid-eye-movement (NREM) stages. This is based on a prior study showing that when people sleep in the REM stage, their respiratory amplitude is smaller than that in the other stages~\cite{respiratory1982}. Hence, we can roughly determine the user's current sleep stage based on the respiratory amplitude. Respiratory amplitude is only an indicator of the division of the sleep stage and we can not regard it as a basis for final judgement.  {However, it serves as an early reference that helps later phases of the sleep stage detection. Normally chest movement amplitude is smaller than abdominal movement amplitude. However, during in different sleep stages, respiration amplitudes differ~\cite{respiratory}. Hence, it is likely that there is a situation where chest movement amplitude in the NREM stages is close to the abdominal movement amplitude in the REM sleep stage. As a result, a naive solution of applying a threshold cannot work satisfactorily but we can combine it with the position of hand detected earlier. Through the above steps, we have been able to determine whether the hands are on the chest or abdomen, and then we can go further to determine the extent of breathing according to the degree of up and down motions, from which we can approximately infer the current sleep stage}. Now we take the case of hands on the abdomen as an example.

\begin{figure}[!t]
\centering
      \includegraphics[width=0.67\linewidth]{Figures/watch.pdf}
  \caption{The first figure on the left is the torso coordinate system, and $Y_t$ points north. The middle figure shows the watch coordinate system when the watch an arbitrary position, and the right of the figure shows that the  watch coordinate system after we completed the coordinate system conversion.}\label{fig:watch}
\end{figure}

Even when the hand is placed on the abdomen, there are some minor changes in the exact location of the user's hand, and the intensity of accelerometer fluctuations caused by respiration also varies greatly. Hence, we cannot use the amplitude information to determine true respiratory amplitude directly. This problem is illustrated in Fig.~\ref{fig:breath_ok1}, where (a) and (b) contain triaxial acceleration measurements at different locations of the abdomen during REM sleep stage, and in (c) which consists of acceleration data for NREM stages at the same approximate position as in (a).  {We can see that when the hand is on the abdomen, but the location differs, we cannot directly judge the respiratory amplitude from the amplitude of the three accelerometer axes.}

To solve this problem, we convert the acceleration data from the wristwatch coordinate system into the torso coordinate system. {As breathing results in chest moving up and down, movements along the z-axis in the torso coordinate system can be used to identify respiratory amplitude}. We express the tri-axial acceleration data as $Acc_w$ = [$X_w$, $Y_w$, $Z_w$] in the wristwatch coordinate system and $Acc_t$ = [$X_t$, $Y_t$, $Z_t$] in the torso coordinate system, as shown in Fig.~\ref{fig:watch}.  {Our coordinate alignment aims to find a rotation matrix $R$ that aligns the watch's coordinate system with the torso coordinate system ({[$X_t$, $Y_t$, $Z_t$]}). Matrix $R$ can be obtained from the three-axis direction information of the orientation sensor. Specifically, we have:}
\begin{equation}
      X_t  = (X_w {\cos\gamma} + Y_w{\sin\gamma}){\cos\theta} + (Y_w\cos\sigma + Y_w\sin\sigma)\sin\theta,
\end{equation}
\begin{equation}
      Y_t = -((Y_w\cos\sigma + Y_w\sin\sigma)\cos\theta - (X_w\cos\gamma + Y_w\sin\gamma)\sin\theta),
\end{equation}
\begin{equation}
      Z_t = (Z_w\cos\gamma - Z_w\sin\gamma)\cos\sigma - (Z_w\cos\gamma - Z_w\sin\gamma)\sin\sigma,
\end{equation}
where $\theta$, $\sigma$ and $\gamma$ are the x, y and z axis data of the orientation sensor respectively, representing the direction angle, the tilt angle and the roll angle collected from the orientation sensor. After alignment, we can see in Fig.~\ref{fig:cordi} that the z-axis shows a periodic signal with significant fluctuations, {while the x- and y-axis data undergo smaller changes around zero, which is consistent with stable sleep influenced only by respiratory patterns.}

%The two graphs on the left of Fig.~\ref{fig:cordi} show the same acceleration data as has been used in (a) and (c) ofFig.~\ref{fig:breath_ok1}, respectively.
The first graph of Fig.~\ref{cordi0} and Fig.~\ref{cordi1} show the same acceleration data as has been used in (a) and (c) of Fig.~\ref{fig:breath_ok1}, respectively. The two right-most graphs correspond to data after coordinate system alignment.  We can see that, prior to alignment, we cannot effectively distinguish the respiratory amplitude of REM and NREM stages from the acceleration amplitude. After coordinate alignment, the respiratory amplitudes are clearly visible from the z-axis data. {To separate REM and NREM stages,} we calculate the variance of z-axis acceleration and use it as a feature to measure the intensity of the fluctuation in a signal, with larger variance corresponding to greater breath amplitude. Note that we cannot use the sum or magnitude of the z-axis as a measure of intensity as the measurements remain affected by gravity.

To summarize, we use respiratory amplitude to detect when the user is in the REM stage.  We calculate respiratory amplitude when the hand is found to be placed on the abdomen or the chest. {Note that as {\systemname} operates using a wrist worn device, it can only detect respiratory events when the hand is placed in one of these two positions. As a measure of respiratory amplitude, we use} the variance of z-axis acceleration. We then use a kNN classifier to find from our training examples, which training example is most similar to the variance of the acceleration collected from z directions. The similarity is measured by calculating the distance on the feature space. We then use the label (either REM or NREM) associated with the nearest training example as the classification outcome.

\begin{figure*}[!t]
	\centering
	\subfigure[REM]{\label{cordi0}
		\includegraphics[width=0.48\linewidth]{Figures/cordi0.pdf}}
	\subfigure[NREM]{\label{cordi1}
		\includegraphics[width=0.48\linewidth]{Figures/cordi.pdf}}
	\caption{Acceleration data for different sleep stages.}\label{fig:cordi}
\end{figure*}

\subsection{Body Rollover Counts \label{sec:bodyrollover}}

Normally, people roll their body around 20-45 times a night. The main function of body rollovers is simply to
maintain a comfortable sleeping position as maintaining the same position for a prolonged period will result in muscular tension due to the hindered blood supply, which can also lead to local numbness~\cite{rollover2014}. Hence, body rollovers are another key indicator about sleep quality.   {{\systemname} detects the number of body rollovers, which provides cues about sleep quality, and roll-over frequency which can help to classify current sleep stage~\cite{rollover2007}}. In general, there are six cases: four posture transition cases between the supine posture and lateral (left or right) posture, and two posture transitions between the left lateral posture and right lateral posture. Fig. \ref{fig:BodyRollover} shows the case when the body moves from the left side to the right side.
The most intuitive way for detecting body rollover events is to estimate the rotation direction of the arm based on the
rotation angle given by the gyroscope. However, doing so is non-trivial because different users can exhibit drastically different patterns for arm rotations; and the subtle changes in the starting arm position for the same sleeping posture could lead to a misprediction. As an alternative to the rotation angle, we find that the tilt angles to be useful for this task because they strongly correlate to a body rollover event. This correlation thus enables us to effectively translate the change of the tilt angle to a body rollover event to count the occurrence of such events. Specifically, the angle values of three different axes are on the falling edge or rising edge simultaneously during a very short time period. Fig. \ref{fig:LeftToRight} -- Fig. \ref{fig:RightToLeft} shows the value changes under different body rollover cases. To this end, a naive method to detect rollovers would be to rely on changes in angle measurements. However, this method suffers a very large error since other hand movements will also induce a similar change.

\begin{figure*}[!t]
\begin{minipage}[t]{0.31\linewidth}\centering
    \includegraphics[width=0.97\linewidth]{Figures/LeftToRight.pdf}\centering
  \caption{From left lateral posture to right lateral posture.}\label{fig:LeftToRight}
\end{minipage}
\hspace{2pt}
\begin{minipage}[t]{0.31\linewidth}\centering
    \includegraphics[width=0.97\linewidth]{Figures/SupineToLeft.pdf}\centering
  \caption{From supine posture to left lateral posture.}\label{fig:SupineToLeft}
\end{minipage}
\hspace{2pt}
\begin{minipage}[t]{0.31\linewidth}\centering
    \includegraphics[width=0.97\linewidth]{Figures/SupineToRight.pdf}
  \caption{From supine posture to right lateral posture.}\label{fig:SupineToRight}
\end{minipage}
\end{figure*}

\begin{figure*}[!t]
\begin{minipage}[t]{0.31\linewidth}\centering
    \includegraphics[width=0.97\linewidth]{Figures/RightToLeft.pdf}
  \caption{From right lateral posture to left lateral posture.}\label{fig:RightToLeft}
\end{minipage}
\hspace{2pt}
\begin{minipage}[t]{0.31\linewidth}\centering
    \includegraphics[width=0.97\linewidth]{Figures/LeftToSupine.pdf}
  \caption{From  left lateral posture to supine posture.}\label{fig:LeftToSupine}
\end{minipage}
\hspace{2pt}
\begin{minipage}[t]{0.31\linewidth}\centering
    \includegraphics[width=0.97\linewidth]{Figures/RightToSupine.pdf}\centering
  \caption{From right lateral posture to supine posture.}\label{fig:RightToLeft}
\end{minipage}
\end{figure*}

To deal with this challenge, we incorporate the body postures to improve the detection accuracy. As shown in Fig. \ref{fig:BodyRollover}, the body postures are different before and after the rollover. Therefore, after we detect the time when the angle values changes, we term the time as a possible rollover time point. Then we use the sleep posture classification algorithm to determine whether the body postures are the same or not over a period of time before and after this point. If the postures are the same, we exclude this time point, otherwise a body rollover event is found. In other words, a body rollover event is recorded when the posture changes are detected before and after the time point. Note that {\systemname} not only counts the number of rollovers, but also reports the exact nature of the rollover event.


\subsection{Detecting Micro Body Movements \label{sec:microbo}}

Besides major body movement, such as rollovers, there often are involuntary body movements that can affect sleep quality. With the deepening of sleep, limbs are extremely relaxed, and a little stimulus will produce trembling and micro beating. Such behavior is most likely to occur during the deep sleep stage and the REM stage~\cite{ancoli2003role,Jean2000Sleep}. Therefore, by detecting such micro body movements and distinguish them from large body movements can help us to further analyze the user's sleep stage. In this paper, we are interested in the sleep-related body movements including hand moving, arm raising, and body trembling.

One of the challenges in detecting the micro body movements is to cancel the inherent noises brought by the accelerometer.
To cancel the noises, we apply a moving window to the collected accelerometer data points to minimize the impact of the outliers. To determine the size of the moving window, we apply different parameter settings to our training data. We found that a moving widow with a size of 35 gives the best average results on our training set. Therefore, we choose to apply a moving window of 35 sample points to the collected user data and then calculate Root Sum Square (RSS) for the data points within a window:

\begin{equation}
      Rss(t) =\sqrt{(acc_x(t))^{2}+(acc_y(t))^{2}+(acc_z(t))^{2}},
\end{equation}
$acc_x(t)$, $acc_y(t)$ and $acc_z(t)$ represent the accelerometer sample value of x-axis, y-axis and z-axis at time stamp $t$ respectively.


We can obtain the first-order derivative of from the RSS values:
\begin{equation}
      V(t)=Rss(t)-Rss(t-1).
\end{equation}

Eventually, we set the threshold to be $0.03$, which can achieve a satisfactory detection performance. We also observe that the micro movement duration is very short, which lasts less than $2$s. However, in our body rollover experiments, we find that the average movement duration is $3$s, as shown in Fig. \ref{fig:LeftToRight} - Fig. \ref{fig:RightToLeft}.
Therefore, we can divide the body movement events into large and micro movement by detecting the signal duration time.

To distinguish among the micro body movements of hand movement, arm raising and body trembling, we first consider the
duration of the movement. We observe from our training data that an arm rising action typically takes around $1.8$s, while a hand movement and a body trembling take around $1$s. Using the duration of the movement, we can distinguish arm rising from the other two movements. We also find that the body trembling tends to lead to a more drastic change in the accelerometer readings compared to the hand movement. This observation is depicted in Fig. \ref{fig:micro-move} using samples from one of our training users. Based on this observation, we then use the change of accelerometer reading to distinguish between the body trembling and the hand movement. We do so by calculating the first-order derivative of accelerometer data to find out the peak of the acceleration data.  If the peak is great than $1.5$ and the duration of the movement took between $0.8$ and $1.2$s, a body trembling is detected; if the peak is less than $1$ and the duration of the movement
took between $0.8$ and $1.2$s, a hand movement is detected; otherwise, if the duration of the movement took between $1.5$ and $2$s, an arm rising is detected. These thresholds are empirically determined from our training data.


\begin{figure}[!t]
\centering
      \includegraphics[width=0.43\linewidth]{Figures/Micromovement.pdf}
  \caption{Accelerometer readings for micro body movements using samples from one of our training users.}\label{fig:micro-move}
\end{figure}



\subsection{ Detecting Acoustic Events \label{sec:acoustic}}
Acoustic events during sleep, such as snore, cough and somniloquy, can reflect and affect user's sleep quality and physical health. For example, the snore is one of the possible symptoms of cerebral infarction patients.  And long-term snoring can also have a serious impact on health and sleep. It can cause sleep apnea or narcolepsy, a sleeping disorder~\cite{snoring2016,snoring2013}. And when there is a cough, the human cerebral cortex cells are always in an excited state, limiting the depth of sleep, allowing only short sleep between wakefulness, like many other sleep disruptions. {\systemname} can detect these different acoustic events, including snore, cough and somniloquy. \ Classical acoustic algorithms use multi-dimensional features to detect acoustic events from a complex environment~\cite{gu2016sleep}. By contrast, We tailor our design to the problem domain to derive a simpler yet effective acoustic detection algorithm, by exploiting the characteristics of the different events that we target.


\subsubsection*{Acoustic Feature Calculation}
In order to identify different acoustic events accurately, we select the short-term average energy and the zero-crossing rate as two features. The short-term average energy of acoustic signal is computed as:
\begin{equation}
  E_i=\sum\nolimits_{j=-\infty}^{\infty}[x(j)\omega(i-j)]^2=\sum\nolimits_{j=i-(N-1)}^{i}[x(j)\omega(i-j)]^2,
\end{equation}
$N$ is the window length, $x$ is the signal and $\omega$ is the impulse response. As we can see that the short-term energy is the weighted sum of squared frame sample. The short-term energy can be used to distinguish the segment of unvoiced and voiced sound. It also can be used to differentiate the speech segment and noise segment  in  case of relatively high signal to noise ratio (SNR). The zero-crossing rate is computed as:
\begin{equation}
  Z_i = \frac{1}{2}\sum\nolimits_{j=0}^{N}|sgn[x_i(j)]-sgn[x_i(j-1)]|.
\end{equation}
It indicates the number of times, which the acoustic signal waveform passes through the horizontal axis (zero level). We carry out an interesting recognition experiment using the microphone built in smartwatch to detect the sound of people during sleep and effectively identify different acoustic events. We focus on three common acoustic events: snore, cough and somniloquy. Ten volunteers worn the smartwatches during sleep to record the acoustic data. We manually label the data with different acoustic events. Fig. \ref{acoustic} shows the acoustic characteristics of  three events. There are six times snore event, two consecutive cough event and somniloquy event.

\begin{figure*}[!t]
\centering
 \subfigure[Snore of six times.]{\label{snore}
   \includegraphics[width=0.32\linewidth]{Figures/snore.pdf}}
 \subfigure[Two consecutive cough.]{\label{cough}
   \includegraphics[width=0.32\linewidth]{Figures/cough.pdf}}
\subfigure[Somniloquy.]{\label{somniloquy}
     \includegraphics[width=0.32\linewidth]{Figures/somniloquy.pdf}}
\caption{The characteristics of different acoustic events.}\label{acoustic}
\end{figure*}


\subsubsection*{Acoustic Event Recognition}
 At the beginning, the algorithm divides the audio stream into frames with equal durations. Each frame is composed of 256 samples, and its duration is 12 ms. To identify the three common acoustic events, we introduce an acoustic recognition algorithm based on the following key observations:

 \begin{itemize}
\item The time intervals between two signals for different acoustic events are quite different from each other. As shown in Fig.~\ref{snore}, there is a long time interval between two snores. While the time interval between two coughs is very short.  In contrast, the  somniloquy signal is irregular and without periodic property.
 \item The duration of one signal for different acoustic events are different from each other. Fig. \ref{acoustic} shows that the duration of a snore is shorter than the duration of a cough and somniloquy. And in general, the duration of a cough is shorter than the duration of a somniloquy signal.
\item The frequencies for different acoustic events are quite different from each other. Snore event has a continuous signal, while the  cough and somniloquy are sudden events, thus the number of consecutive occurrences is very small.
\end{itemize}
In conclusion, the ``interval'', ``duration'' and ``frequency'' of acoustic events can be used as three unique features. Based on the above three observations, our acoustic event recognition algorithm involves the following two steps. First, in order to estimate the interval and frequency of an acoustic event, we perform the peak detection. We use the short-term average energy to calculate the peak value of the acoustic signal. When the peak exceeds a certain threshold, such as 3 dB in this paper, we record the position of each peak and calculate the interval between two consecutive peaks. Next, we can estimate the frequency of an acoustic event by counting the number of peaks over a time window. Second, to estimate the duration of the acoustic event, we perform the start-point and end-point detection.

Traditional end-point detection algorithm \cite{stowell2015detection}, however, uses a fixed double threshold and must be obtained by a large number of data samples, which has two drawbacks. First, the fixed double threshold may cause error detection at the beginning of an acoustic event. Second, the requirement of a large number of data samples would lead to large system latency. To address these issues, we develop a detection method that does not require pre-sampling to obtain the optimal thresholds. Instead, our algorithm estimates the thresholds on a per signal basis. This strategy not only reduces the number of data samples needed, but also leads to a higher accuracy.

Specifically, since the first few frames and the last few frames are mostly mute or are the background noise, we select the first five frames and the last five frames to calculate their short-term energy, which are denoted as $E_s$ and $E_e$, respectively. And then the two are combined to obtain the mean $E_n$ as the estimated energy value of the noise segment. Let the maximum value of the short-term energy over all frames denoted as $\max (E)$. Then, the average short-term energy $DE$ is given as:
\begin{eqnarray}
      &E_n = \frac{(E_s+E_e)}{2}, \\
      &DE = \max (E)-E_n.\label{eq:DE}
\end{eqnarray}
So we can use $EH$ and $EL$ to represent the high and low threshold of short-term energy as:
\begin{eqnarray}
      &EH=\alpha \times DE+E_n,\\
      &EL=\beta \times DE+E_n,
\end{eqnarray}
where $\alpha$ and $\beta$ are the multiplier factors of the energy value $DE$.


Here, we need to choose the right values for $\alpha$ and $\beta$ to ensure that we can accurately detect the start and end points of a speech signal. To that end, we use the night time sound data from our training dataset to determine $\alpha$ and $\beta$. Specifically, we tested $\alpha$ with values ranging between 0.1 and 0.5, and $\beta$ with values ranging from 0.01 to 0.09. We found that setting $\alpha$ to  0.1 and $\beta$ to  0.06 gives the best overall results in our training dataset.

To ensure sudden noise does not interfere with detection results, we set the minimum length of the signal segment and count the length of the signal during the search for the start and end point. Finally, if the signal length is less than the set minimum, it is considered as a noise segment. The results of the start-point and end-point detection are shown in Fig. \ref{acoustic}, we calculate the length of each speech segment and calculate their averages as the duration of the acoustic event. Last, we count the number of peak points to see whether it meets the third key observation or not.



\subsection{Tracking Illumination Conditions \label{sec:illumination}}
Studies have shown that there is a significant interaction between the illuminance level and the mental state of the individual \cite{light77}. For example, the bright light can counteract subjective fatigue during the daytime, but at night it will seriously affect the sleep quality. Too much light exposure can shift our biological clock, which makes restful sleep difficult to achieve, it affects our sleep and wake cycle \cite{light2007}.  Besides, we also note that the dim light will affect our sleep too. According to the study \cite{light2016}, it can be learned that the dim artificial light during sleep is significantly associated with the general increase in fatigue, and the proper light can be used to increase the sense of exhaustion and promote sleep. So the illumination condition in a sleeping environment has a significant influence on sleep. {\systemname} use the ambient light sensor to detect the illumination condition during sleep. We visit the bedroom of our ten testing users at night and the use of the ambient light sensor to test the lighting conditions in the bedroom. We find that in the absence of lights in the bedroom, the light sensor reading is between 1 Lux to 4 Lux. In some cases, the light of the smartwatch screen can raise the light sensor reading to 4 Lux when the bedroom is dark. In other cases, when the bedroom has weak lights (e.g., when the bedroom is illuminated using a table lamp), the light sensor's average readings are below 10 Lux. Based on these observations, we divide the illumination intensity of the bedroom into two categories. When the bedroom has weak lights (i.e., the light sensor reading is no greater than 10 Lux), and when the bedroom has strong lights (i.e., the light sensor reading is greater than 10 Lux). Using the threshold of 10 Lux, we can then map the light sensor readings to one of these two groups.


However, the light sensor may be obscured, which leads to large errors in measuring the illumination level. For example, a user's smartwatch may be covered under the quilt when he/she turned over unconsciously, and thus the illumination readings on the smartwatch may not reflect the real lighting situation. Most of the previous works on smartphone-based light detection have used the proximity sensor to detect whether the light sensor is blocked or not. However, such an approach is not applicable to the off-the-shelf smartwatches because they typically do not have a proximity sensor.  To deal with this practical challenge, the key is that the illumination would drop suddenly when the smartwatch is covered by other objects. There are two possibilities for the sudden drop in light intensity. For most smartwatches, the light sensors are usually installed in the front face of it. The first case is the indoor lighting facilities are closed. The second case is the wrist turned so that the back of the hand become downward, thus blocking the light sensor in front of the smartwatch. Such a situation often happens in real life. For examples, when a user changes the sleeping posture to the left side, his/her hand may be close to the pillow with the palm facing up; or the back of the hand may become downward because of a hand movement. To avoid this erroneous illumination condition measurement, we detect whether the user is performing a wrist flip over a period of time during the intensity plummeting. We detect the wrist flip based on two aspects: (i) the rotation angle of the smartwatch; (ii) whether the light intensity maintains stable after the sharp drop. If the wrist flips, we use the average of the previous light intensity as the intensity of the time period. It should be noted that, because the illumination condition detection algorithm is relatively simple, it is not explained in the experimental part.

\subsection{Sleep Stage and Quality}

Sleep is generally considered as a cyclical physiological process composed of three stages: rapid eye movement (REM) stage, light sleep stage and deep sleep stage. REM is an active period of sleep marked by intense brain activities and dream occurrence. Light sleep stage is a period of relaxation, when the heartbeat, breathing rate and muscle activity slow down. Deep sleep stage triggers hormones to promote body growth, as well as the repair and restoration of energy.  The biological characteristics of different sleep stages exhibit distinguishingly. In the clinical sleep study, the sleep stages are mainly identified by simultaneously evaluating three fundamental measurement modalities including brain activities, eye movements, and muscle contractions. The EEG measure using electrodes placed around the scalp interpret various sleep/wake states of the brain. And, EMG and EOG using electrodes placed on the skin near the eyes and on the muscles, respectively, measures in deeply differentiating REM stage from all the other stages. But, apart from the implicit physiological activities, sleepers usually exhibit distinguishable physical activities in different sleep stages. For example, there are somniloquy and body trembles caused by frequent dreams generally appear in REM, large body movements such as body rollovers and arm raising happen in light sleep and micro body movements such as body trembling and snoring occur in deep sleep.  In the meanwhile, the breathing amplitude in NREM stage is larger compared with the REM stage. Moreover, the sleep cycle usually repeats four to six times over a night. The sleeper usually experiences a transition from light sleep to deep sleep and then enters REM, but sometimes there is also possible a phenomenon of skipping some certain sleep stages occurs during sleep. However, despite this, the dependence between two successive sleep stages still exists and different sleep stages have potential conversion probabilities, which also mentioned in Sleep Hunter \cite{gu2016sleep}.

To separate between these states, we build a Hidden Markov Model~\cite{johnson2010hidden} for identifying the current sleep stage of the user. As the input, i.e., the observed states, we use a series of detected sleep events and the sleep stage sequence is modelled as a hidden state, i.e., $obs_t={NB(t),NB_M(t),BA(t),NA(t)}$ represent the feature vector at the detection phase $t$. The explanation of each item, which is the input of HMM, is listed as follows. $NB(t)$: the number of occurrences of body rollover during the detection phase t. $NB_M(t)$: the number of occurrences of micro body movement. $BA(t)$: the measurement of respiratory amplitude.  $NA(t)$: the number of occurrences of the acoustic events. And $states_t$ =$\{$light sleep; deep sleep; REM$\}$ is an output of our model, which represents the sleep stage in the detection phase $t$. In the training of the HMM model, we use nocturnal sleep data from 10 volunteers who participated in the training parameters of each algorithm and make their sleep-related events as observation sequences and the corresponding sleep stages detected by Fitbit as hidden state sequences, to generate HMM models. Specifically, we first use the maximum likelihood estimation method for parameter estimation, the state transition matrix and the confusion matrix, and then use the Viterbi algorithm to acquire a series of implicit state sequences corresponding to observed sequence.  As a result, we can estimate the sleep stage during a time window. Finally, we can get the durations of all sleep stages over the whole sleep process.

Further, to quantize the sleep quality, we use the Sleep Quality Report model introduced in \cite{gu2016sleep}. Let $SQ$ be the value of the sleep quality, then $SQ$ is given as follow:
 \begin{equation}
SQ=\frac{(REM \times 0.5+Light \times 0.75+Deep) \times 100}{REM+Light+Deep}
 \end{equation}
where, REM, Deep and Light represent the duration time in a sleep process. The range of $SQ$ is from 50 to 100. A high value of $SQ$ shows a better  sleep quality.

\section{EVALUATION METHODOLOGY}\label{sec:expsetup}


\begin{figure}[!t]
	\centering
	\includegraphics[width=0.57\linewidth]{Figures/setup.pdf}
	\caption{Experimental setup in one of our participants' home. }\label{fig:setup}
\end{figure}

\subsection{{Pilot Study: Training Data}}\label{sec:trainingdata}

Prior to our main study, we carried out a pilot study that was used to inform our algorithm design, and to provide training data for the
algorithms integrated into {\systemname}. Our pilot study consisted of two groups (aged ranged from 15 to 60). One group   {consisted of}
randomly selected 100 volunteers to conduct questionnaire surveys to provide the basis for our algorithm design. The other group consisted
of 10 users whose data was used to train our models.


To improve the effectiveness of the algorithms integrated into {\systemname}, especially sleep posture and hand position detection, we
elicited questionnaires to 100 volunteers to identify their common sleep posture. The main content of the questionnaire was about their
common arm position in the four basic sleeping postures. Based on this investigation and previous
research~\cite{position2014,HandPosition2}, we selected the positions to consider in {\systemname}. We also found that these arm positions
are representative during the training and testing of our algorithms.

To train the models used in our system and to determine optimal parameter values, a small-scale pilot study with $10$ participants was
carried out prior to the main experiment. The training examples used to train our algorithms and to determine the algorithm parameters are
collected from 10 users (5 males and 5 females). Our testing users were asked to wear a smartwatch to sleep and collected the sensor data
while they were sleeping. Every testing user contributes 10 nocturnal sleep data over a two-week period. These users are different from
those taking part in our evaluation (Sec.~\ref{sec:evalusers}).


\subsection{Evaluation Setup\label{sec:evalusers}}

\cparagraph{Participants.} We evaluate {\systemname} through experiments conducted in $15$ single-occupancy homes over a two-week period.
The participants include 6 males and 9 females, whose age spans 15 to 60 years. To ensure little sleep had no effect on the results, each
participant was required to sleep at least $6$ hours per night during the study period. Two of our participants have been diagnosed with
long-term, on-going sleep-related disorders, and one participant has described that his sleep is significantly affected by snoring. The
remaining participants reported their sleep quality to go up and down. The study was approved by local IRB, and participants were
separately asked to consent to release their data for analysis. In total, we collected 210 sets of nocturnal sleep data from our
participants.

\cparagraph{Setup.} During the study, participants are asked to wear a smartwatch on their wrist. To obtain ground truth of sleep events,
three video cameras were placed on the ceiling to monitor the user's sleep activities, as shown in Fig.~\ref{fig:setup}. The cameras have
night vision and thus can accurately capture the sleep activity in dark.


\cparagraph{Data collection and annotation.} The recorded video footage was manually labeled with different sleep activities and the labels
were used as ground truth in our evaluation. Specifically, we consider the respiratory amplitude during the NREM stage as large amplitude,
and the one during REM stage as normal amplitude. For the acquisition of sleep stage information, we confirm labels when both Fitbit and
{\systemname} reach a consensus. To demonstrate the overall benefits of {\systemname} and the events captured by it, we separately
collected ground truth information about sleep quality using questionnaires which were administrated each morning. The questionnaires were
based on the Pittsburgh Sleep Quality Index (PSQI), a widely used and validated questionnaire in sleep quality
research~\cite{buysse1989pittsburgh}. The results of the user survey are presented at Sec.~\ref{sec:user_survey}. Finally, we collected the
sleep stage estimations given by a Fitbit Charge2 and use them as ground truth for sleep stage estimation. While the performance of Fitbit
is not comparable to medical-grade PSG\footnote{Equipping the participants with PSG was not feasible as it would disrupt their normal
sleeping routines and potentially bias and reduce sleep activities, which are the main focus of our work. Moreover, the goal of our
experiments is not to demonstrate that {\systemname} is capable of medical-grade sleep monitoring, but to demonstrate that it performs
comparably to commercial systems in common sleep monitoring tasks, while at the same time being able to capture a much richer set of sleep
information.}, it has been shown to have a good association in adults~\cite{evenson2015systematic,fitbit01,fitbit02,fitbit03}, especially
in estimating REM and light sleep stages.


\cparagraph{Competitive schemes.} In addition to Fitbit Charge2, we also compare our approach against Sleep Hunter~\cite{gu2016sleep}, a
state-of-the-art mobile-based sleep monitoring approach, and a smartphone-based sleep monitoring app named Sleep as
Android~\cite{SleepAndroid}. The former app is designed to estimate sleep time and assess sleep by recording the state of motions and the
number of body exercises. The latter focuses on estimating sleep stages and evaluate the sleep quality using the tracked sleep-related
events. To provide a fair comparison against these baselines, we also place a smartphone next to the user's body on the bed to collect the
data for Sleep Hunter and Sleep as Android.
	





\subsection{Prototype Implementation \label{sec:implementation}}
We prototype and evaluate {\systemname} on a Huawei Smartwatch 2 wearable device. The smartwatch is equipped with a Quad-core Cortex-A7
processor at 1.1 GHz. It runs the Android Wear 2.0 operating system. We use five sensors of the smartwatch: the accelerometer, gyroscope,
microphone, the light and the orientation sensors. To reduce the energy consumption of the smartwatch, in the experiments we analyze the
sensor data on a XiaoMI Note2 Android smartphone to which the smartwatch sends sensor measurements over Bluetooth. The sensors on the
smartwatch are sampled every $30$ milliseconds, which was chosen to balance between information quality and energy consumption.
{\systemname} starts tracking sleep events when it detects that the light is off (which can also be triggered by the user during daytime)
and there has been nobody movement for 30 minutes. As part of an initialization process, {\systemname} estimates the initial body posture
and hand position. It then uses these as a starting point to monitor sleep events like the body posture, rollovers, hand positions and body
movements.

\section{RESULTS}\label{sec:4experiment}
In this section, we detail the evaluation results for our system.

\subsection{Evaluation of Subcomponents}
We focus on the detection accuracy about five events, that are body posture, the body rollover, the hand position, the micro body movements and the acoustic events.

\subsubsection{Sleep Posture Classification Performance}
\label{subsub:bodyposture}

We first test the overall classification performance of different body postures. The ground truth of body postures is recorded by the cameras. To avoid biases in the evaluation, and to assess the generalization performance of our approach, we consider a cross-validation scheme where all data from a single participant is used for training and data from the remaining $14$ participants is used for testing. The motivation for using data from a single user as training data is to highlight the capability of {\systemname} to accurately characterize body posture with very little training data, while at the same time being able to generalize across users. The final performance is then calculated as the averaged accuracy across the 15 folds; as shown in Fig.~\ref{fig:posture_zhu} and Table \ref{tab:posture}. We can observe that the posture detection accuracy is consistently high across all users, and does not show major variations across users. This good performance benefits from the distinct characteristics of arm position under different sleeping postures. Compared to results reported for SleepMonitor~\cite{sleepmonitor}, {\systemname} consistently improves performance which is mainly due to the template-based classifier that we use to verify classifications of the prone and supine states. In particular, {\systemname} achieves around $5$ percentage units higher performance on the prone state than SleepMonitor and overall has a lower false positive rate. In addition, {\systemname} considers more hand positions than SleepMonitor in 4 sleeping postures. In terms of errors, due to angular characteristics of acceleration being similar between the supine posture with the hand putting on the head and the left-lateral posture, a small amount of the supine postures are classified as left lateral. From the results, we can also observe that the total amount of the prone posture is smaller than the number of other postures, which suggests that people are not accustomed to sleeping in this position because it is neither healthy nor comfortable.

\begin{figure}
	\centering
	\begin{minipage}{.5\textwidth}
	 \centering
	\includegraphics[width=0.95\linewidth]{Figures/posture_zhu.pdf}
	\caption{Detection accuracy of body postures.}\label{fig:posture_zhu}
	\end{minipage}%
	\begin{minipage}{.5\textwidth}
			\centering
		\includegraphics[width=0.95\linewidth]{Figures/handposition_zhu.pdf}
		\caption{Identification accuracy of hand positions.}\label{fig:hand_zhu}
	\end{minipage}
\end{figure}

\begin{table}[!t]\footnotesize
	\centering
	\renewcommand\arraystretch{0.35}
	\caption{The confusion matrix of body posture classification.}\label{tab:posture}
	\begin{tabular}{c| r | r | r | r | r | r}
		\cline{1-7}
		&\multicolumn{1}{ c|}{ }
		& \multicolumn{4}{ c|}{ }\\
		\multirow{2}*{}
		&\multicolumn{1}{c|}{\multirow{2}*{{\textbf{Result}}}}
		&\multicolumn{4}{c|}{{\textbf{Prediction}}}
		& \multirow{4}*{{\textbf{Recall}}} \\
		\cline{3-6}
		& & & & & \\
		\multicolumn{1}{c|}{{}}
		&  \multicolumn{1}{c|}{{}}
		&  \multicolumn{1}{l|}{{Supine}}
		&  \multicolumn{1}{l|}{{Left Lateral}}
		&  \multicolumn{1}{l|}{{Right Lateral}}
		&  \multicolumn{1}{l|}{{Prone}}   \\
		& & & & & \\
		\cline{1-7}
		& & & & & \\
		\multirow{5}{*}{\begin{sideways}{{Groundtruth}}\end{sideways}}
		&   {Supine}   & {\bf{{1182}}}    &   $25$      &   $4$      &   $9$    &   {96.7\%}\\
		& & & & & \\
		\cline{2-7}
		& & & & & \\
		&   {Left Lateral}   &   $6$      &   {\bf{{1292}}}     &   $0$      &   $0$   &   {99.5\%} \\
		& & & & & \\
		\cline{2-7}
		& & & & & \\
		&   {Right Lateral}   &   $7$      &   $0$      &  {\bf{{1275}}}      &   $12$  &   {98.5\%}  \\
		& & & & & \\
		\cline{2-7}
		& & & & & \\
		&   {Prone}   &   $19$      &   $2$      &   $3$      &   {\bf{{567}}}   &   {95.9\%} \\
		& & & & & \\
		\cline{1-7}
		& & & & & \\
		&   {Precision}    &   {97.3 \%}   &   {98.0\%}   &   {99.5\%}   &   {96.4\%}    \\
		& & & & & \\
		\cline{1-7}
	\end{tabular}
\end{table}


\subsubsection{Performance of Body Rollover Counting}
To verify the efficiency of body rollover detection algorithm, we compare each user's body rollover events detected by {\systemname} against the data labeled by watching the video. The performance is shown in Table \ref{tab:rollver}. We can see that User 3, User 4 and User 13 have an unusually high number of rollovers. For User 3 and User 4, they have difficulty in falling asleep due to the sleep disorder. User 13 needs to rollover frequently because of his loudly snoring. As we demonstrate in Sec.~\ref{sec:user_survey}, these participants also suffered from poor sleep quality and hence indicate how the information extracted by {\systemname} can support the detection of sleep problems. For all the 15 users, the detection accuracies are all very high, and the lowest one is still 87\%. Thus {\systemname} can accurately distinguish the large hand movement from the body rollover in bed. Moreover, detecting errors in body rollover events will not have a significant impact on our end result, because the division of sleep stages is a comprehensive consideration of all the detected features in each stage, such as micro body movement and acoustic events.

\begin{table}[!thbp]\footnotesize
  \caption{Detection accuracy of body rollover.}\label{tab:rollver}
  \setlength{\tabcolsep}{3pt}
\renewcommand{\arraystretch}{0.67}{\multirowsetup}{\centering}
        \begin{tabular}{lccccccccccccccc}
        \toprule
         \textbf{Testing User ID}    & 1& 2  & 3& 4& 5& 6& 7& 8& 9& 10& 11& 12& 13& 14& 15\\
        \midrule
            \rowcolor{Gray} {Labeled \#body-rollover}  &231&204&442&397&198&101&196&164&193&208&131&205&342&149&156 \\
                 {Accuracy} &91\%& 94\% &88\%&93\%&96\%&94\%&87\%&90\% &93\% &94\% &92\% &94\% &89\% &90\% &95\%\\
        \bottomrule
 \end{tabular}
\end{table}

\subsubsection{Performance of Hand Position Recognition}
To test the recognition performance of different hand positions, we consider the same cross-validation scheme used for body posture detection, i.e., one user's data is used for training the classifier and the remaining 14 users' data as test data. The classifier for detecting the hand movement trajectory is combined with the detection of periodic signals caused by respiration, then the hand position on the chest (or abdomen or head) can be identified. In our dataset, 14\%, 36\% and 22\% of the time the hand in the supine posture during sleep were placed on the head, abdomen, and chest respectively. Fig.~\ref{fig:hand_zhu} illustrates the accuracy of hand position across 15 users. As we can see that with just one set of training data, the accuracies for different users are all higher than 87\%. Therefore, our system can achieve a good identification accuracy for different hand positions. Moreover, we find that at least four out of fifteen participants tend to put their hands on their heads; one participate unconsciously puts his hand on his chest which makes him have nightmares. Those are all bad habits disrupting a good sleep. {\systemname} can report such key findings to improve the users' sleep qualities.


\subsubsection{Performance of Micro Body Movement Detection}

To assess the detection accuracy of micro body movements, we manually label the ground truth recorded by the camera during sleep, including hand moving, arm raising, and body trembling. We also use the accelerometer embedded in the smartphone which placed on the bed to record the occurrence of micro body movements, so as to avoid missing some movements such as trembling concealed by the duvet. For the acceleration data collected by smartphone, we first smooth the acceleration along the three axes, calculate Root Sum Square (RSS) to merge them and obtain the first-order derivative of the merged acceleration. And then we use the threshold detection method to mark the occurrence of motion. Since body trembling is the easiest to be covered, we only focus on such events. So we use a smartphone to detect the occurrence of events and the classification of the event is not performed. Table~\ref{tab:micro_move} list the total number of three micro body movements for each user over the testing period of 14 days, and Fig.~\ref{fig:micro_movement_zhu} reports the accuracy of {\systemname} for detecting these micro body movements. It shows that the accuracies for all users are very close, that is, there will be no major changes between users. And from Fig. \ref{fig:micro_combine}, we find that even though the worst classification result belongs to the hand movement, the average precision value and recall value still exceed 75\%. The averaged accuracies of arm raising and body trembling are 93\% and 84\%, respectively. Because the training data volume for the hand movement and body trembling is small, so the performance can be improved by setting each user a threshold by collecting a longer term's sleeping data. In addition, the purpose of micro body movement detection is to detect different sleep stages, and the hand movement usually appears in all sleep stages, thus the poor accuracy of hand moving does not have a significant impact on the final result.


\begin{figure*}
	\centering
	\begin{minipage}{.485\textwidth}
		 \includegraphics[width=0.95\textwidth]{Figures/micro_movement_zhu.pdf}
		\caption{Micro body movement detection accuracy for each user.}\label{fig:micro_movement_zhu}	
	\end{minipage}%
\hspace{3pt}
	\begin{minipage}{.485\textwidth}
	 \centering
	\includegraphics[width=7.8cm,height=5cm]{Figures/micro_combine1.pdf}
	\caption{Average precision and recall for micro body movement detection.}\label{fig:micro_combine}
	\end{minipage}
\end{figure*}


\begin{table}[!t]\footnotesize
  \caption{The number of micro body movements per user.}\label{tab:micro_move}
   \renewcommand\arraystretch{1}{\multirowsetup}{\centering}
        \begin{tabular}{lccccccccccccccc}
        \toprule
         \textbf{Testing User ID}    & 1& 2  & 3& 4& 5& 6& 7& 8& 9& 10& 11& 12& 13& 14& 15\\
        \midrule
            \rowcolor{Gray} {Labeled \#hand movement}  &52&49&67&55&78&65&59&70&61&53&81&55&60&59&63 \\
             { Labeled \#arm raising} &48&50&62&53&66&49&57&50&73&45&54&69&57&56&61\\
             \rowcolor{Gray} { Labeled \#body trembling} &28&32&25&29&34&25&20&30&24&26&27&35&24&22&25\\
        \bottomrule
 \end{tabular}
\end{table}


 \subsubsection{Performance of Acoustic Events Detection}

To study the detection accuracies of different acoustic events, we compare the ground truth recorded by the camera with the detected results by our system. Table~\ref{tab:sound} shows the results across 15 participants. We can see that the precision for the cough event is 88.9\%, which is slightly lower than for the other three event types. The reason is that different user's cough patterns are different, the pre-defined parameters in the detection model do not include all possible patterns. For example, some people have a fast and continuous pattern of coughing, while others have a slower intermittent pattern. In fact, we train these parameters, namely the "interval", "duration" and "frequency" of acoustic events, with only 120 sets of nighttime sound data. Those data come from 40 (21 males and 19 females) volunteers of different ages (from 15 to 60 years old) who are prone to snoring, coughing, or somniloquy at night. To further improve the detection accuracy, we can train particular parameters for different users. And we can further expand the training data to include more possible patterns, and can also make reasonable estimates of the possible patterns to refine the range of parameters and thus increase the accuracy.


\begin{table}[!t]\footnotesize
  \centering
 \renewcommand\arraystretch{0.35}
  \caption{The confusion matrix of acoustic events detection.}\label{tab:sound}
\begin{tabular}{c| r | r | r | r | r | r}
   \hline
   &\multicolumn{1}{ c|}{ }
   & \multicolumn{4}{ c|}{ }\\
   \multirow{2}*{}
&\multicolumn{1}{c|}{\multirow{2}*{{ \textbf{Result}}}}
&\multicolumn{4}{c|}{{ \textbf{Prediction}}}
& \multirow{4}*{{ \textbf{Recall}}} \\
    \cline{3-6}
    & & & & & \\
    \multicolumn{1}{c|}{{}}
    &  \multicolumn{1}{c|}{{}}
    &  \multicolumn{1}{l|}{{ Snore}}
    &  \multicolumn{1}{l|}{{ Cough}}
    &  \multicolumn{1}{l|}{{ Somniloquy}}
    &  \multicolumn{1}{l|}{{ Other}}   \\
    & & & & & \\
     \cline{1-7}
    & & & & & \\
    \multirow{5}{*}{\begin{sideways}{{ Groundtruth}}\end{sideways}}
    &   { Snore}   & {\bf{{96}}}    &   $0$      &   $0$      &   $9$    &   {91.4\%}\\
    & & & & & \\
    \cline{2-7}
    & & & & & \\
   &   { Cough}   &   $3$      &   {\bf{{64}}}     &   $0$      &   $4$   &   {90.1\%} \\
    & & & & & \\
     \cline{2-7}
    & & & & & \\
    &   { Somniloquy}   &   $0$      &   $3$      &  {\bf{{42}}}      &   $2$  &   {89.4\%}  \\
    & & & & & \\
     \cline{2-7}
    & & & & & \\
    &   { Other}   &   $0$      &   $5$      &   $4$      &   {\bf{{325}}}   &   {97.3\%} \\
    & & & & & \\
    \hline
    & & & & & \\
    &   { Precision}      &   {96.9\%}   &   {88.9\%}   &   {91.3\%}   &   {95.6\%}    \\
    & & & & & \\
    \hline
   \end{tabular}
\end{table}


\subsection{Overall Performance \label{sec:overall_per}}

\subsubsection{Performance of Sleep Stage Detection}

In order to prove that the detected events not only reflect the user's sleep habits, but also effectively identify the sleep stages to assess the sleep quality, we regard the reported results from Fitbit Charge2 as the ground truth. To perform the evaluation, we randomly choose $50$ sets of sleep data from the data so that at least $3$ sets per participant are chosen for evaluation. For detecting changes in sleep stage, {\systemname} uses event-driven detection. When there is no sleep event detected in 15 minutes, we evaluate the sleep stage. When an event occurs, we immediately evaluate the sleep stage and use this time as the starting point for the next 15 minutes. The averaged precision value and recall value are shown in Table~\ref{tab:sleep stage}. It indicates that though {\systemname} may make misjudgment between the light sleep and REM, overall the performance is satisfying and comparable to current consumer-grade monitors. Moreover, as we later demonstrate, the main benefits of {\systemname} result from its capability to estimate a wide range of sleep events and how they relate to sleep quality, not from its performance in sleep stage detection where medical PSG measurements are required for accurate assessment of sleep stages.

\begin{table}[!t]\footnotesize
%\setlength{\tabcolsep}{1pt}
\renewcommand{\arraystretch}{0.55}{\centering}
	\caption{{The confusion matrix of sleep stage detection.}}\label{tab:sleep stage}
	\begin{tabular}{c| r | r | r | r | r}
		\hline
		&\multicolumn{1}{ c|}{ }
		& \multicolumn{3}{ c|}{ }\\
		\multirow{2}*{}
		&\multicolumn{1}{c|}{\multirow{2}*{{ \textbf{Result}}}}
		&\multicolumn{3}{c|}{{ \textbf{Prediction}}}
		& \multirow{3}*{{ \textbf{Recall}}} \\
		%&\multicolumn{5}{ c |}{\textbf{\small Prediction}} \\
		% & \multicolumn{5}{ c |}{ } \\
		\cline{3-5}
		& & & & & \\
		\multicolumn{1}{c|}{{}}
		&  \multicolumn{1}{c|}{{}}
		&  \multicolumn{1}{l|}{{ REM}}
		&  \multicolumn{1}{l|}{{ Light Sleep}}
		&  \multicolumn{1}{l|}{{ Deep Sleep}} \\
		\cline{1-6}
		& & & & & \\
		\multirow{1}{*}{\begin{sideways}{{Groundtruth}}\end{sideways}}
		&   { REM}   & {\bf{{476}}}    &   $143$      &   $61$     &   {70.0\%}\\
		& & & & & \\
		\cline{2-6}
		& & & & & \\
		&   { Light Sleep}   &   $131$      &   {\bf{{508}}}     &   $91$      &   {69.6\%} \\
		& & & & & \\
		\cline{2-6}
		& & & & & \\
		&   { Deep Sleep}   &   $63$      &   $113$      &  {\bf{{262}}}      &   {59.8\%}  \\
		& & & & & \\
		\cline{1-6}
		& & & & & \\
		&   { Precision}      &   {71.0\%}   &   {66.5\%}   &   {63.3\%}   \\
		& & & & & \\
		\hline
	\end{tabular}
\end{table}

\subsubsection{Effect of Respiratory Amplitude on Sleep Stage Detection}

When we detect different sleep stages, we also consider the respiratory amplitude when the hand's position is in the abdomen or chest. To assess the effectiveness of respiratory amplitude estimation, we evaluate the performance of the sleep stage detection in two cases, that is with and without taking the respiration amplitude into account. The performance of sleep stage detection is shown in Table~\ref{tab:respiratory}. For three different sleep stages, both the precision and recall values are improved with the help of respiratory amplitude estimation. In fact, the respiratory frequency can also be used as a feature to help us to detect sleep stages. But in fact their essence the same. The difference in respiratory amplitude will also affect the difference in respiratory frequency, because when the respiratory amplitude is large, the time taken for one breath will be long, and the frequency of breathing will be slower. In {\systemname}, we choose the respiratory amplitude because the feature is very intuitive.

\begin{table}[!t]\footnotesize
	\centering
	\renewcommand\arraystretch{0.3}
	\caption{Effect of respiratory amplitude estimation.}\label{tab:respiratory}
	\begin{tabular}{l| l | r | r | r | r | r | r |}
		\cline{2-8}
		&\multicolumn{1}{ c|}{ }
		&\multicolumn{2}{ c|}{ }
		&\multicolumn{2}{ c|}{ }
		& \multicolumn{2}{ c|}{ }\\
		%  \multirow{4}*{}
		&\multicolumn{1}{c|}{}
		&\multicolumn{2}{c|}{\textbf{\footnotesize REM}}
		&\multicolumn{2}{c|}{\textbf{\footnotesize Light Sleep}}
		&\multicolumn{2}{c|}{\textbf{\footnotesize Deep Sleep}} \\
		%&\multicolumn{5}{ c |}{\textbf{\small Prediction}} \\
		% & \multicolumn{5}{ c |}{ } \\
		\cline{2-8}
		& & & & & & &\\
		\multicolumn{1}{c|}{\textbf{}}
		&  \multicolumn{1}{l|}{\textbf{Features}}
		&  \multicolumn{1}{l|}{\footnotesize Precision}
		&  \multicolumn{1}{c|}{\footnotesize Recall}
		&  \multicolumn{1}{c|}{\footnotesize Precision}
		&  \multicolumn{1}{c|}{\footnotesize Recall}
		&  \multicolumn{1}{c|}{\footnotesize Precision}
		&  \multicolumn{1}{c|}{\footnotesize Recall}\\
		& & & & & & &\\
		\cline{2-8}
		& & & & & & &\\
		\multirow{5}{*}
		&   \textbf{\footnotesize Without Respiratory Amplitude}   & $62.9\%$    &   $63.4\%$      &   $59.4\%$      &   $63.9\%$    &   $57.7\%$ &  $54.1\%$ \\
		& & & & & & &\\
		\cline{2-8}
		& & & & & & &\\
		&   \textbf{\footnotesize With Respiratory Amplitude}   &   $71.0\%$      &   $70.0\%$     &   $66.5\%$      &   $69.7\%$   &   $63.3\%$ &   $59.8\%$ \\
		& & & & & & &\\
		\cline{2-8}
	\end{tabular}
\end{table}

  \begin{table}[!t]\footnotesize
 	\centering
 	\renewcommand\arraystretch{0.3}
 	\caption{Performance of sleep stage detection comparison.}\label{tab:comparison}
 	\begin{tabular}{c| r | r | r | r | r |}
 		\cline{2-6}
 		&\multicolumn{1}{ c|}{ }
 		&\multicolumn{2}{ c|}{ }
 		&\multicolumn{2}{ c|}{ }\\
 		&\multicolumn{1}{c|}{}
 		&\multicolumn{2}{c|}{\textbf{\footnotesize Light Sleep}}
 		&\multicolumn{2}{c|}{\textbf{\footnotesize Deep Sleep}} \\
 		\cline{2-6}
 		\multicolumn{1}{c|}{\textbf{}}
 		&  \multicolumn{1}{l|}{\diagbox{System}{Stage}}
 		&  \multicolumn{1}{c|}{\footnotesize Precision}
 		&  \multicolumn{1}{c|}{\footnotesize Recall}
 		&  \multicolumn{1}{c|}{\footnotesize Precision}
 		&  \multicolumn{1}{c|}{\footnotesize Recall}\\
 		\cline{2-6}
 		& & & & & \\
 		&   \textbf{\footnotesize {\systemname}}   & $66.5\%$    &   $69.6\%$      &   $63.3\%$      &   $59.8\%$  \\
 		& & & & &  \\
 		\cline{2-6}
 		& & & & & \\
 		&   \textbf{\footnotesize Sleep As Android}   &   $27.8\%$      &   $35.4\%$     &   $35.7\%$      &   $50.2\%$   \\
 		& & & & &  \\
 		\cline{2-6}
 		& & & & & \\
 		&   \textbf{\footnotesize Sleep Hunter}   &   $66.7\%$      &   $66.1\%$     &   $60.0\%$      &   $50.7\%$   \\
 		& & & & &  \\
 		\cline{2-6}
 	\end{tabular}
 \end{table}

\subsubsection{Performance Comparison}

We compare {\systemname} with two state-of-the-art sleep monitoring applications. The first is a sleep detection app called ``Sleep As Android", and the second one is a smartphone-based system named Sleep Hunter~\cite{gu2016sleep}. The former app is designed to estimate sleep time and assess sleep by recording the state of motions and the number of body exercises. The latter focuses on estimating sleep stages and evaluate the sleep quality using the tracked sleep-related events. Considering that Sleep As Android can only detect light sleep stage and deep sleep stage, we only compare the performance of these two stages. Table \ref{tab:comparison} shows the detection results. As we can see, {\systemname} significantly outperforms Sleep As Android and deliver better performance than Sleep Hunter. The performance advantage of {\systemname} comes from the incorporation of rich and complicated sleep events.

\begin{table}[!t]\footnotesize
 	\centering
 	\renewcommand\arraystretch{0.3}
 	\caption{Smartwatch based acoustic event detection comparison.}\label{tab:compare_sound1}
 	\begin{tabular}{c| r | r | r | r | r | r | r |}
 		\cline{2-8}
 		&\multicolumn{1}{ c|}{ }
 		&\multicolumn{2}{ c|}{ }
 		&\multicolumn{2}{ c|}{ }
         &\multicolumn{2}{ c|}{ }\\
 		&\multicolumn{1}{c|}{}
 		&\multicolumn{2}{l|}{\textbf{\footnotesize Snore}}
        &\multicolumn{2}{l|}{\textbf{\footnotesize Cough}}
 		&\multicolumn{2}{l|}{\textbf{\footnotesize Somniloquy}} \\
 		\cline{2-8}
 		\multicolumn{1}{c|}{\textbf{}}
 		&  \multicolumn{1}{c|}{\diagbox{System}{Event}}
 		&  \multicolumn{1}{c|}{\footnotesize Precision}
 		&  \multicolumn{1}{c|}{\footnotesize Recall}
        &  \multicolumn{1}{c|}{\footnotesize Precision}
 		&  \multicolumn{1}{c|}{\footnotesize Recall}
 		&  \multicolumn{1}{c|}{\footnotesize Precision}
 		&  \multicolumn{1}{c|}{\footnotesize Recall}\\
 		\cline{2-8}
 		& & & & & & &\\
 		&   \textbf{\footnotesize {\systemname}}   & $96.9\%$    &   $91.4\%$      &   $88.9\%$      &   $90.1\%$  &   $91.3\%$  &   $89.4\%$ \\
 		& & & & & & & \\
 		\cline{2-8}
 		& & & & & & &\\
 		&   \textbf{\footnotesize Sleep Hunter}   &   $71.0\%$      &   $73.0\%$     &   $71.0\%$      &   $63.0\%$   &   $89.0\%$  &   $83.5\%$\\
 		& & & & & & & \\
 		\cline{2-8}
 	\end{tabular}
 \end{table}

\begin{table}[!t]\footnotesize
 	\centering
 	\renewcommand\arraystretch{0.3}
 	\caption{Smartwatch based sleep stage detection comparison.}\label{tab:compare_stage1}
 	\begin{tabular}{c| c | c | c | c | c | c | c |}
 		\cline{2-8}
 		&\multicolumn{1}{ c|}{ }
 		&\multicolumn{2}{ c|}{ }
 		&\multicolumn{2}{ c|}{ }
        &\multicolumn{2}{ c|}{ }\\
 		&\multicolumn{1}{c|}{}
 		&\multicolumn{2}{c|}{\textbf{\footnotesize REM}}
        &\multicolumn{2}{c|}{\textbf{\footnotesize Light Sleep}}
 		&\multicolumn{2}{c|}{\textbf{\footnotesize Deep Sleep}} \\
 		\cline{2-8}
 		\multicolumn{1}{c|}{\textbf{}}
 		&  \multicolumn{1}{c|}{\diagbox{System}{Stage}}
 		&  \multicolumn{1}{c|}{\footnotesize Precision}
 		&  \multicolumn{1}{c|}{\footnotesize Recall}
        &  \multicolumn{1}{c|}{\footnotesize Precision}
 		&  \multicolumn{1}{c|}{\footnotesize Recall}
 		&  \multicolumn{1}{c|}{\footnotesize Precision}
 		&  \multicolumn{1}{c|}{\footnotesize Recall}\\
 		\cline{2-8}
 		& & & & & & &\\
 		&   \textbf{\footnotesize {\systemname}}   & $71.0\%$    &   $70.0\%$      &   $66.5\%$      &   $69.6\%$  &   $63.3\%$  &   $59.8\%$ \\
 		& & & & & & & \\
 		\cline{2-8}
 		& & & & & & &\\
 		&   \textbf{\footnotesize Sleep Hunter}   &   $61.2\%$      &   $67.6\%$     &   $61.5\%$      &   $60.2\%$   &   $56.8\%$  &   $47.9\%$\\
 		& & & & & & & \\
 		\cline{2-8}
 	\end{tabular}
 \end{table}



We also implement the algorithms employed by Sleep Hunter and apply them to the data collected using our smartwatch. This experiment allows us to check if the better performance of {\systemname} is due to the use of a smartwatch instead of a mobile phone.

For body movement detection, applying the algorithms employed by Sleep Hunter to our smartwatch data gives a comparable accuracy of around 96\% when the system only identify between drastic and small body movements. However, Sleep Hunter doesn't work when we need to detect finer-grained body movements.  {\systemname} outperforms Sleep Hunter by delivering an accuracy of drastic body movements around 90\% (see Table~\ref{tab:rollver}) and an accuracy of finer-grained body movements more than 78\% (see Fig.~\ref{fig:micro_combine}). For acoustic event detections, we apply the Sleep Hunter algorithms to detect snore, cough and somniloquy. The results in Table \ref{tab:compare_sound1} suggest that {\systemname} gives better performance over Sleep Hunter in detecting these acoustic events. Finally, we apply the sleep stage detection model used by Sleep Hunter to combine sleep-related events to identify sleep stages. The results are shown in Table \ref{tab:compare_stage1}. Again,  {\systemname} outperforms the Sleep Hunter model with a higher accuracy and recall.


This experiment confirms that the algorithms used by Sleep Hunter for sleep event and stage detection are not tuned for the smartwatch. Compared to Sleep Hunter, {\systemname} can detect sleep events and stages with a higher accuracy using a set of carefully designed methods target to target smartwatch.


\subsubsection{User Survey}\label{sec:user_survey}

To understand and verify how the additional information captured by {\systemname} supports users, at the end of the experiments the participants are administrated a survey to the participants asking about their experiences with {\systemname} and their personal sleeping patterns. We combine these results with the subjective sleep quality estimates obtained through the PSQI questionnaires administered during the study. We considered two groups of users in our survey. As the main source of information, we consider the $15$ participants in our experiments who were asked about their experiences with {\systemname}, their subjective sleep quality assessment, and details of their personal sleep patterns. This set of users was augmented with external participants who were asked about their interested in the events that {\systemname} is capable of detecting. The questions in our survey include:
\begin{enumerate}
  \item Subjective sleep quality (5-levels, 1 for excellent and 5 for worst),
  \item Sleep duration,
  \item Sleep disturbances,
  \item Daytime dysfunction.
\end{enumerate}
For the above four items, each one is rated on a 1 to 5 scale. These scores are first summed to yield a total score, which ranges from 0 to 20. Then we merge every five neighboring scores into one scale and eventually divide the total scores into four levels, recorded as 0, 1, 2 and 3, representing poor, general, good and excellent, respectively. This step is necessary to compare the results of the user survey against the sleep quality estimation provided by {\systemname} and Fitbit.

\begin{table} \footnotesize
\setlength{\tabcolsep}{1pt}
\renewcommand{\arraystretch}{0.9}{\multirowsetup}{\centering}
\caption{{Results of sleep quality assessments. The first three rows show the sleep quality scores of the different systems (mean and standard deviation) for each user across $14$ days, whereas the last two rows  compare sleep quality labels between subjective assessments and those returned by {\systemname} and FitBit. }}\label{tab:quality}
\begin{tabularx}{\textwidth}{X cccccccccccccccc }
        \toprule
         \textbf{User ID} & 1 & 2 & 3 & 4 & 5 & 6 & 7 & 8 & 9 & 10 & 11 & 12 & 13 & 14 & 15\\
         \midrule
         \rowcolor{Gray}{\systemname} & 3 (1.8) & 3 (1.5) & 0 (1.0) & 1 (1.8) &  2 (1.2) &  2 (1.7) &  3 (1.0) & 0 (1.3) &  2 (1.3) &  2 (1.0) & 2 (0) & 2 (1.5) &  1 (1.8) &  0 (1.0) &  2 (1.3)\\
         Fitbit & 3 (1.7) & 3 (2.2) & 0 (1.3) & 0 (1.8) & 1 (2.2) & 3 (1.8) & 2 (1.9) & 3 (1.8) & 2 (1.7) & 2 (1.9) & 2 (1.3) & 2 (2.0) & 2 (2.3) & 1 (1.8) & 2 (1.7) \\
         \rowcolor{Gray} User score & 3 (0.0) & 2 (1.4) & 0 (0.0) &  0 (1.7) & 2 (1.7) &  2 (1.0) & 3 (1.0) & 1 (1.7) & 1 (1.4) & 2 (1.0) &  2 (1.0) & 3 (2.0) & 0 (1.8) & 0 (1.0) & 2 (1.0)\\
         {\systemname} & P&O&P&O&P&P&P&O&O&P&P&O&O&P&P\\
         \rowcolor{Gray} Fitbit & P&O&P&P&O&O&O&\textbf{B}&O&P&P&O&\textbf{B}&O&P\\
         \bottomrule
\end{tabularx}
\end{table}

In Table \ref{tab:quality}, we show the mean sleep quality score across 14 days given by each user together with the estimations produced by {\systemname} and Fitbit. We also give the standard deviation of the scores across the 14-day period per user. This number is given in the brackets next to the mean score. The last two rows in Table~\ref{tab:quality} compare the estimation given by {\systemname} and Fitbit against the user self-rating score. In these two specific rows, a label of `P' indicates an estimated score perfectly matches the user score, a label of `O' means the estimation error is within one scale point (for example, {\systemname} rates the user sleep to be excellent while the user's self-rating is good), and a label of `B' indicates the estimation error is greater than a scale point. As can be seen from the table, the estimation given by {\systemname} is more likely to match the user's self-rating compared to Fitbit (as indicated by having more `P' labels - 9 vs 6 ) and, unlike Fitbit, the estimation error given by {\systemname} is never greater than one scale point. To further validate this, we calculated the Spearman $\rho$-correlation~\cite{richardson2015nonparametric} between the user scores and each of the two systems, {\systemname} and Fitbit. {\systemname} provides higher correlation coefficient ($\rho = 0.842$) than Fitbit ($\rho = 0.500$). The difference in correlation was found statistically significant using a one-tailed test carried out through a Fisher r-z transformation ($Z = 1.66, p < 0.05$).  In summary, {\systemname} thus gives a better sleep quality assessment compared to Fitbit in our evaluation.

While the results above demonstrate that {\systemname} is capable of accurately estimating sleep quality, the main benefit from {\systemname} compared to previous works is not its sleep quality performance but its capability to analyze and capture the {\em root cause} of sleep issues. To demonstrate this, we carried out a follow-up analysis where we examined the events captured by {\systemname} for each of the six users assessing their sleep quality negatively (poor or general subjective quality, i.e., rating 0 or 1 in Table~\ref{tab:quality}). For four of the six users, we were able to find clear causes for their poor sleep. For one of the users, {\systemname} indicated difficulties in falling asleep, which was reflected in a high body rollover count. Further analysis of captured events indicated ambient noise and lighting to be most likely reasons for this participant. Another user complained of feeling of numbness in the arm after sleep. Events captured by {\systemname} showed that this was likely due to bad hand posture as the person tended to put the hand on top of the head before sleeping. The third user complained of frequent nightmares. Analysis of {\systemname} events showed that the person habitually slept on the left side, which has been shown to have the higher risk on nightmares~\cite{nightmare}, and often placed a hand on top of the chest, which creates additional pressure and can lead to nightmares. Finally, one of the users mentioned suffering from long-term snoring problems, which we also were able to detect from the events captured by {\systemname}. The events also highlighted that the person was often sleeping in the supine position, which further increases susceptibility to snoring-related problems. Existing systems are only capable of capturing some of these factors influencing sleep quality and thus they are not capable of providing a holistic view of the participant's sleep quality, whereas {\systemname} is capable of providing very detailed information about sleep events. To further demonstrate the benefits of {\systemname} compared to previous works, we asked the 15 participants to make appropriate adjustments according to our recommendations and to conduct a return visit survey three weeks later. It was found that some of the users were able to reduce symptoms and to improve their average quality of sleep based on the suggestions.
	
As for the user experience, results from the survey highlighted a strong interest in the information captured by {\systemname}. In particular, 80\% of participants believe that the detection of sleep posture is very necessary, showing their sleep posture can not only help people to avoid health problems caused by long-term improper sleeping posture, but also help us find out the reasons for the next day's physical discomfort, such as dizziness, muscle soreness may be due to improper sleeping posture. And there are some users are troubled by snoring. This may be due to improper sleeping posture. We map the detected snoring event and sleeping posture to suggest the user to modify his posture to a suitable posture to reduce the harm caused by long-term snoring. 60\% of the participants thought it useful to detect the hand position in supine posture, even one user mentioned that he did often have nightmares and our system found his hand was often placed on his chest, and then {\systemname} could remind him that he should take some measures to avoid such a position and thus reduce the poor sleep quality that nightmare brings. Only 20\% of participants found it useful to calculate the number of body rollover. However, detection of rollovers is useful in segmenting sleep stages. Furthermore, body rollover counts could be used to derive additional information to the user, such as how restless or peaceful the sleep has been overall.

\input{5discussion.tex}
\section{Related Work}\label{sec:5related}

%Poor sleep can lead to numerous diseases, such as  endocrine dyscrasia, depression, immunity decline
%\cite{vgontzas2009insomnia,gottlieb2005association}. Thus, a lot of research works have been proposed to monitor the sleep
%\cite{langkvist2012sleep,hao2013isleep,bai2012will,kay2012lullaby,bain2003evaluation,pombo2016ubisleep}.

There is an extensive body work on sleep monitoring and
tracking~\cite{langkvist2012sleep,hao2013isleep,bai2012will,kay2012lullaby,bain2003evaluation,pombo2016ubisleep}. We summarize some of the
most relevant work in this section.

\subsection{Medical Grade Sleep Monitoring Solutions}
Traditionally, the dedicated medical technologies, like EEG, ECG and EMG \cite{saper2005hypothalamic}, have been applied for sleep
monitoring. Those technologies rely on the certain biomedical signals, such as brain wave, muscle tone and eye movement, to assess the
sleep quality. For example, the EEG technology in \cite{langkvist2012sleep,oropesa1999sleep,ebrahimi2008automatic} monitors the brain
waves, and then recognizes the sleep stages by leveraging unsupervised learning approaches. Although a high accuracy can be achieved by
those technologies, they have two drawbacks. First, those  technologies require the dedicated medical devices, which are expensive compared
the widely available smartwatch or smartphone. Second, they require the users to be attached many sensors on the human body, which may
cause a healthy person hard to sleep and result in large errors. Compared with those medical technologies, our system has two advantages.
First, we only need a smartwatch, which is more cost effective. Second, the smartwatch has little disruption to a user's normal sleep, thus
we can monitor the user's sleep quality more precisely.

\subsection{Smartphone-Based Approaches}
Numerous approaches have been proposed to exploit the use of a smartphone for sleep monitoring. iSleep \cite{hao2013isleep} measures the
sleep quality by recording sleep-related acoustic events and evaluates the sleep quality using the Pittsburgh Sleep Quality Index (PSQI)
\cite{carpenter1998psychometric}. Bai \etal~\cite{bai2012will} use a wide range of sensor data captured by the smartphone sensors,
including the accelerometer, gyroscope, and microphone, to predict a user's sleep quality. The work presented in \cite{kay2012lullaby}
leverages the smartphone sensors to record the sleep disruptor for a user, while the work presented in \cite{choe2011opportunities}
explores a series of opportunities to support healthy sleep behaviors. Other approaches predict the sleep quality by leveraging the
smartphone to monitor the external factors, such as the daily activity, the sleeping environment and location, and family settings
\cite{chen2013unobtrusive,zhang2013real}. In addition to the aforementioned approaches, there is a wide range of smartphone base sleep
monitoring applications. Examples of such applications include Sleep As Android \cite{SleepAndroid}, Sleep Journal \cite{SleepJournal}, and
YawnLog \cite{YawnLog}.

Those smartphone-based systems, however, require placing the smartphone at a specifically location near to the user, which may not always
be feasible. For example, the work presented in \cite{gu2016sleep} requires the smartphone to be placed next to the user's head, and to
remain stationary throughout the sleeping process. Such a constraint is hard to satisfy because of body movements during the sleep.
Furthermore, prior research also shows that many users do not want to place their mobile phone too close to the body due to health risk
concerns~\cite{StepHealth,Quorasleep}.

Unlike existing smartphone-based solutions, \systemname uses the commodity smartwatch for sleep monitoring. Since many users are willing to
wear a smartwatch throughout the sleep, the smartwatch can remain relatively close to the user body. This allows us to collect a wider
range of sleep-relevant events with a higher accuracy. This richer set of data thus leads to better sleep monitoring and quality
assessment.

\subsection{Wearable-Device-Based Approaches}
 With the widely use of wearable devices, more and more researchers and industries try to use the smartwatch or wearable-wrist for sleep
 monitoring \cite{bain2003evaluation,bonnet2003insomnia,pombo2016ubisleep,caviness1996myoclonus}. The Sleeptracker \cite{sleeptracker} is a
 wristwatch-shaped unit that apart from telling the time, also infers whether the user is in deep sleep, light sleep, or awake, using an
 accelerometer. The ubiSleep \cite{pombo2016ubisleep} joints heart rate, accelerometer, and sound signals collected into the smartwatch for
 noninvasive sleep monitoring. The aXbo alarm clock [9] is packaged as a stand-alone application in the form of an alarm clock that
 wirelessly communicates with a wrist-band unit. The Zeo \cite{caviness1996myoclonus} is similarly using an alarm clock base unit with a
 worn sensing device, but the latter is a headband rather than a wrist-band that based on electroencephalograph (EEG) to monitor sleep. We
 know Zeo has good performance in sleep stage detection compared to some wristband sleep monitoring products because products based on some
 biological signals like EEG are able to get more accurate and more representative sleep data than actigraphy-based \cite{Actigraphy} sleep
 monitoring products. However, the vast majority of biosignal-based sleep detection approaches require specialized equipment, which is high
 cost and complex to operate, while the approaches based on actigraphy are well
adapted and user-friendly to accept and understand.

Moreover, these actigraphy-based wristband devices or smartwatch Apps only can gather coarse-grained information and  do not design a model
for deep understanding the relationship between a user's sleep pattern and the sensor data. For example, many smartwatch Apps, like Jawbone
Up \cite{Jawbone}, FitBit \cite{fitbit}, YawnLog \cite{YawnLog} and WakeMate \cite{WakeMate}, do not show how they assess a user's sleep
quality based on what kind of sensor data. In addition,  there is some work that uses wearable devices to detect certain detailed events of
sleep. The work presented in \cite{wear_related2} detects roll-over and measure sleep quality using a wearable sensor. The approach
presented in \cite{wear_related3} uses a single chest-worn sensor to extract body acceleration and sleep position changes.


Compared with the existing smartwatch or wearable-wrist based systems, our system is a more complete sleep monitoring system, and is based
only on sensors in commercial smart watches without additional hardware. It collect an extensive set of sleep-related events, many of which
are not supported in prior work. And for sleep monitoring in daily life, it is more practical and will not invasion users' normal sleep ,
and more and more people are willing to accept to wear watches to fall asleep, unlike other wearable devices such as chest-worn sensors,
most people are still unwilling to accept to wear her to sleep. Moreover, our original intention and focus are more inclined to enable
users to have a deeper and more comprehensive understanding of their sleep, explore the causes of sleep quality, and provide users with
more practical advice to point them in a clear direction for improving sleep quality and being healthy.

In Table \ref{tab:function}, we compare {\systemname}'s functionalities with eight other existing sleep monitoring systems for mobile and
wearable devices. As can be seen from the table, {\systemname} supports the detection for the largest number of sleep related events, and
thus provides a rich set of sleep data.

%
\begin{table*}[!t]\footnotesize
 \setlength{\tabcolsep}{3pt}
\renewcommand{\arraystretch}{0.8}{\multirowsetup}{\centering}
  \caption{Compare the supporting functionalities of mobile based sleep monitoring systems.\label{tab:function}}
  \begin{tabular}{l cccccc}
        \toprule
       % \multirow{2}{*}{\textbf{System}} & \multicolumn{6}{c}{\textbf{Detected Events}} \\
%        \cline{2-7}
        & Heart Rate & Acoustic & Postures & Body Movements & Hand Positions & Sleep Stages \\
        \midrule
       \rowcolor{Gray} \systemname & &\checkmark&\checkmark&\checkmark&\checkmark&\checkmark\\
       Sleep as Andriod &  & \checkmark & & & & \checkmark \\
       \rowcolor{Gray} Sleep Hunter & & \checkmark &  & \checkmark & & \checkmark \\
       Sleep Monitor & &  & \checkmark &  & &  \\ %~\cite{sleepmonitor}
       \rowcolor{Gray} Sleeptracker & \checkmark &  &  & & & \checkmark \\ %~\cite{sleeptracker}
       isleep  && \checkmark & & \checkmark & & \\ %~\cite{hao2013isleep}
       \rowcolor{Gray} Fitbit & \checkmark &  & &  & & \checkmark \\
       Jawbone& &  & &  & & \checkmark \\ %~\cite{Jawbone}
       \rowcolor{Gray}  ubiSleep & \checkmark & \checkmark & &  & & \\ %~\cite{pombo2016ubisleep}
        \bottomrule
 \end{tabular}
\end{table*}



\subsection{Summary of Prior Work}
To summarize, as we can see from Table \ref{tab:related_work}, the advantage of {\systemname} is that it can detect more fine-grained sleep-related events to obtain more abundant sleep information, which is currently on the market for commercial or scientific research sleep monitoring system can not be achieved. And the performance of {\systemname} has been improved to some extent. Our original intention and focus are more inclined to enable users to have a deeper and more comprehensive understanding of their sleep, explore the cause s of sleep quality, and provide users with more practical advice to point them in a clear direction for improving sleep quality and being healthy. And compared with some medical technology like PSG, our advantages are inexpensive and easy to deploy at home, so it is suitable for most general public. And it does not need a large number of instruments attached to the user's body, thus it has less intrusiveness for sleep and does not require professional personnel to operate. Although the accuracy of {\systemname}'s ability to detect sleep cannot be compared to medical technology, it is enough for the average family's daily sleep monitoring needs. The most important is that we concentrates on physical activities rather than biomedical signals, so these rich physical activities detected are easily understood by users, and they can be adjusted with improved and improved sleep based on the results of monitoring.


\begin{table*}[!t]\footnotesize
 \setlength{\tabcolsep}{2.7pt}
\renewcommand{\arraystretch}{0.8}{\multirowsetup}{\centering}
  \caption{Summary of existing solutions.}\label{tab:related_work}
        \begin{tabular}{lcccccc}
        \toprule
        {System} & {High accuracy} & {Practicability} & {Low disruptive} & {Low cost} & {Informativeness} & {Interpretability}  \\
        \midrule
        \rowcolor{Gray} PSG     &  $\checkmark$ & &  &   & $\checkmark$ &  \\

        Smartphone& &$\checkmark$ &$\checkmark$  &$\checkmark$   & & $\checkmark$ \\

        \rowcolor{Gray} Wearable& &$\checkmark$ & $\checkmark$ & $\checkmark$  & & $\checkmark$ \\
        {\systemname} & &$\checkmark$ &$\checkmark$  & $\checkmark$  & $\checkmark$&$\checkmark$  \\

        \bottomrule
  \end{tabular}
\end{table*}

\input{7conclusions}

\bibliographystyle{ACM-Reference-Format}
\bibliography{sleep_ref}

\end{document}
