\section{EVALUATION METHODOLOGY}\label{sec:expsetup}


\begin{figure}[!t]
	\centering
	\includegraphics[width=0.57\linewidth]{Figures/setup.pdf}
	\caption{Experimental setup in one of our participants' home. }\label{fig:setup}
\end{figure}

\subsection{{Pilot Study: Training Data}}\label{sec:trainingdata}

Prior to our main study, we carried out a pilot study that was used to inform our algorithm design, and to provide training data for the
algorithms integrated into {\systemname}. Our pilot study consisted of two groups (aged ranged from 15 to 60). One group   {consisted of}
randomly selected 100 volunteers to conduct questionnaire surveys to provide the basis for our algorithm design. The other group consisted
of 10 users whose data was used to train our models.


To improve the effectiveness of the algorithms integrated into {\systemname}, especially sleep posture and hand position detection, we
elicited questionnaires to 100 volunteers to identify their common sleep posture. The main content of the questionnaire was about their
common arm position in the four basic sleeping postures. Based on this investigation and previous
research~\cite{position2014,HandPosition2}, we selected the positions to consider in {\systemname}. We also found that these arm positions
are representative during the training and testing of our algorithms.

To train the models used in our system and to determine optimal parameter values, a small-scale pilot study with $10$ participants was
carried out prior to the main experiment. The training examples used to train our algorithms and to determine the algorithm parameters are
collected from 10 users (5 males and 5 females). Our testing users were asked to wear a smartwatch to sleep and collected the sensor data
while they were sleeping. Every testing user contributes 10 nocturnal sleep data over a two-week period. These users are different from
those taking part in our evaluation (Sec.~\ref{sec:evalusers}).


\subsection{Evaluation Setup\label{sec:evalusers}}

\cparagraph{Participants.} We evaluate {\systemname} through experiments conducted in $15$ single-occupancy homes over a two-week period.
The participants include 6 males and 9 females, whose age spans 15 to 60 years. To ensure little sleep had no effect on the results, each
participant was required to sleep at least $6$ hours per night during the study period. Two of our participants have been diagnosed with
long-term, on-going sleep-related disorders, and one participant has described that his sleep is significantly affected by snoring. The
remaining participants reported their sleep quality to go up and down. The study was approved by local IRB, and participants were
separately asked to consent to release their data for analysis. In total, we collected 210 sets of nocturnal sleep data from our
participants.

\cparagraph{Setup.} During the study, participants are asked to wear a smartwatch on their wrist. To obtain ground truth of sleep events,
three video cameras were placed on the ceiling to monitor the user's sleep activities, as shown in Fig.~\ref{fig:setup}. The cameras have
night vision and thus can accurately capture the sleep activity in dark.


\cparagraph{Data collection and annotation.} The recorded video footage was manually labeled with different sleep activities and the labels
were used as ground truth in our evaluation. Specifically, we consider the respiratory amplitude during the NREM stage as large amplitude,
and the one during REM stage as normal amplitude. For the acquisition of sleep stage information, we confirm labels when both Fitbit and
{\systemname} reach a consensus. To demonstrate the overall benefits of {\systemname} and the events captured by it, we separately
collected ground truth information about sleep quality using questionnaires which were administrated each morning. The questionnaires were
based on the Pittsburgh Sleep Quality Index (PSQI), a widely used and validated questionnaire in sleep quality
research~\cite{buysse1989pittsburgh}. The results of the user survey are presented at Sec.~\ref{sec:user_survey}. Finally, we collected the
sleep stage estimations given by a Fitbit Charge2 and use them as ground truth for sleep stage estimation. While the performance of Fitbit
is not comparable to medical-grade PSG\footnote{Equipping the participants with PSG was not feasible as it would disrupt their normal
sleeping routines and potentially bias and reduce sleep activities, which are the main focus of our work. Moreover, the goal of our
experiments is not to demonstrate that {\systemname} is capable of medical-grade sleep monitoring, but to demonstrate that it performs
comparably to commercial systems in common sleep monitoring tasks, while at the same time being able to capture a much richer set of sleep
information.}, it has been shown to have a good association in adults~\cite{evenson2015systematic,fitbit01,fitbit02,fitbit03}, especially
in estimating REM and light sleep stages.


\cparagraph{Competitive schemes.} In addition to Fitbit Charge2, we also compare our approach against Sleep Hunter~\cite{gu2016sleep}, a
state-of-the-art mobile-based sleep monitoring approach, and a smartphone-based sleep monitoring app named Sleep as
Android~\cite{SleepAndroid}. The former app is designed to estimate sleep time and assess sleep by recording the state of motions and the
number of body exercises. The latter focuses on estimating sleep stages and evaluate the sleep quality using the tracked sleep-related
events. To provide a fair comparison against these baselines, we also place a smartphone next to the user's body on the bed to collect the
data for Sleep Hunter and Sleep as Android.
	





\subsection{Prototype Implementation \label{sec:implementation}}
We prototype and evaluate {\systemname} on a Huawei Smartwatch 2 wearable device. The smartwatch is equipped with a Quad-core Cortex-A7
processor at 1.1 GHz. It runs the Android Wear 2.0 operating system. We use five sensors of the smartwatch: the accelerometer, gyroscope,
microphone, the light and the orientation sensors. To reduce the energy consumption of the smartwatch, in the experiments we analyze the
sensor data on a XiaoMI Note2 Android smartphone to which the smartwatch sends sensor measurements over Bluetooth. The sensors on the
smartwatch are sampled every $30$ milliseconds, which was chosen to balance between information quality and energy consumption.
{\systemname} starts tracking sleep events when it detects that the light is off (which can also be triggered by the user during daytime)
and there has been nobody movement for 30 minutes. As part of an initialization process, {\systemname} estimates the initial body posture
and hand position. It then uses these as a starting point to monitor sleep events like the body posture, rollovers, hand positions and body
movements.
