\section{Related Work}\label{sec:5related}

%Poor sleep can lead to numerous diseases, such as  endocrine dyscrasia, depression, immunity decline
%\cite{vgontzas2009insomnia,gottlieb2005association}. Thus, a lot of research works have been proposed to monitor the sleep
%\cite{langkvist2012sleep,hao2013isleep,bai2012will,kay2012lullaby,bain2003evaluation,pombo2016ubisleep}.

There is an extensive body work on sleep monitoring and
tracking~\cite{langkvist2012sleep,hao2013isleep,bai2012will,kay2012lullaby,bain2003evaluation,pombo2016ubisleep}. We summarize some of the
most relevant work in this section.

\subsection{Medical Grade Sleep Monitoring Solutions}
Traditionally, the dedicated medical technologies, like EEG, ECG and EMG \cite{saper2005hypothalamic}, have been applied for sleep
monitoring. Those technologies rely on the certain biomedical signals, such as brain wave, muscle tone and eye movement, to assess the
sleep quality. For example, the EEG technology in \cite{langkvist2012sleep,oropesa1999sleep,ebrahimi2008automatic} monitors the brain
waves, and then recognizes the sleep stages by leveraging unsupervised learning approaches. Although a high accuracy can be achieved by
those technologies, they have two drawbacks. First, those technologies require the dedicated medical devices, which are expensive compared
the widely available smartwatch or smartphone. Second, they require the users to be attached many sensors on the human body, which may
cause a healthy person hard to sleep and result in large errors. Compared with those medical technologies, our system has two advantages.
First, we only need a smartwatch, which is more cost effective. Second, the smartwatch has little disruption to a user's normal sleep, thus
we can monitor the user's sleep quality more precisely.

\subsection{Smartphone-Based Approaches}
Numerous approaches have been proposed to exploit the use of a smartphone for sleep monitoring. iSleep \cite{hao2013isleep} measures the
sleep quality by recording sleep-related acoustic events and evaluates the sleep quality using the Pittsburgh Sleep Quality Index (PSQI)
\cite{carpenter1998psychometric}. Bai \etal~\cite{bai2012will} use a wide range of sensor data captured by the smartphone sensors,
including the accelerometer, gyroscope, and microphone, to predict a user's sleep quality. The work presented in \cite{kay2012lullaby}
leverages the smartphone sensors to record the sleep disruptor for a user, while the work presented in \cite{choe2011opportunities}
explores a series of opportunities to support healthy sleep behaviors. Other approaches predict the sleep quality by leveraging the
smartphone to monitor the external factors, such as the daily activity, the sleeping environment and location, and family settings
\cite{chen2013unobtrusive,zhang2013real}. In addition to the aforementioned approaches, there is a wide range of smartphone base sleep
monitoring applications. Examples of such applications include Sleep As Android \cite{SleepAndroid}, Sleep Journal \cite{SleepJournal}, and
YawnLog \cite{YawnLog}.

The aforementioned smartphone-based systems, however, require placing the smartphone at a specifical location near to the user, which may
not always be feasible. For example, the work presented in \cite{gu2016sleep} requires the smartphone to be placed next to the user's head,
and to remain stationary throughout the sleeping process. Such a constraint is hard to satisfy because of body movements during the sleep.
Furthermore, prior research also shows that many users do not want to place their mobile phone too close to the body due to health risk
concerns~\cite{StepHealth,Quorasleep}.

Unlike existing smartphone-based solutions, \systemname uses the commodity smartwatch for sleep monitoring. Since many users are willing to
wear a smartwatch throughout the sleep, the smartwatch can remain relatively close to the user body. This allows us to collect a wider
range of sleep-relevant events with a higher accuracy. This richer set of data thus leads to better sleep monitoring and quality
assessment.

\subsection{Wearable-Device-Based Approaches}
Some of the more recent works have exploited smartwatch or wearable-wrist for sleep
 monitoring \cite{bain2003evaluation,bonnet2003insomnia,pombo2016ubisleep,caviness1996myoclonus}. The Sleeptracker \cite{sleeptracker} uses the accelerometer data to infer the user's sleep stages. The ubiSleep \cite{pombo2016ubisleep} joints heart rate, accelerometer, and sound signals collected from the smartwatch for
 sleep monitoring. The Zeo \cite{caviness1996myoclonus} uses a smartwatch together a headband to monitor sleep. The data collected by the headband helps Zeo to achieve good tracking precisions, but wearing a headband can disrupt a user's normal sleep.

The majority of the prior smartwatch-based sleep monitoring systems only gather coarse-grained information but do not provide analysis to
help users to understand the correlation between sleep quality and sleep-related events. For examples, many smartwatch Apps, including
Jawbone Up \cite{Jawbone}, FitBit \cite{fitbit}, YawnLog \cite{YawnLog} and WakeMate \cite{WakeMate}, only provide the result of sleep
quality assessments but there is little information on how the sleep quality is evaluated. Without such information, it is difficult for a
user to understand the root cause of poor sleep. Therefore, these systems offer little help in assisting users to improve their sleep.In
addition to sleep monitoring, other works use wearable devices to detect specific sleep events, such as roll-over~\cite{wear_related2}, or
body acceleration and sleep position changes~\cite{wear_related3}. However, they only support a limited set of sleep events and thus just
scratch the surface of what could be possibly done on smartwatches.



In Table \ref{tab:function}, we compare {\systemname}'s functionalities with eight other existing sleep monitoring systems for mobile and
wearable devices. As can be seen from the table, {\systemname} supports the detection for the largest number of sleep related events, and
it does so without requiring additional hardware. We have shown that this extensive set of sleep events not only leads to more accurate
sleep quality assessments, but also enables users to better understand the cause of poor sleep.



%
\begin{table*}[!t]\footnotesize
 \setlength{\tabcolsep}{3pt}
\renewcommand{\arraystretch}{0.8}{\multirowsetup}{\centering}
  \caption{Compare the supporting functionalities of mobile based sleep monitoring systems.\label{tab:function}}
   \vspace{-2mm}
  \begin{tabular}{l cccccc}
        \toprule
       % \multirow{2}{*}{\textbf{System}} & \multicolumn{6}{c}{\textbf{Detected Events}} \\
%        \cline{2-7}
        & \textbf{Heart rate} & \textbf{Acoustic events} & \textbf{Postures} &\textbf{ Body movements} & \textbf{Hand positions} & \textbf{Sleep stages} \\
        \midrule
       \rowcolor{Gray} \textbf{\systemname} & &\checkmark&\checkmark&\checkmark&\checkmark&\checkmark\\
       Sleep as Andriod &  & \checkmark & & & & \checkmark \\
       \rowcolor{Gray} Sleep Hunter & & \checkmark &  & \checkmark & & \checkmark \\
       Sleep Monitor & &  & \checkmark &  & &  \\ %~\cite{sleepmonitor}
       \rowcolor{Gray} Sleeptracker & \checkmark &  &  & & & \checkmark \\ %~\cite{sleeptracker}
       isleep  && \checkmark & & \checkmark & & \\ %~\cite{hao2013isleep}
       \rowcolor{Gray} Fitbit & \checkmark &  & &  & & \checkmark \\
       Jawbone& &  & &  & & \checkmark \\ %~\cite{Jawbone}
       \rowcolor{Gray}  ubiSleep & \checkmark & \checkmark & &  & & \\ %~\cite{pombo2016ubisleep}
        \bottomrule
 \end{tabular}
\end{table*}



\subsection{Summary of Prior Work}
To summarize, as we can see from Table \ref{tab:related_work}, the advantage of {\systemname} is that it can detect more fine-grained
sleep-related events to obtain more abundant sleep information, for which the current commercial or scientific research sleep monitoring
system cannot be achieved. Furthermore, the performance of {\systemname} has been improved to some extent. Our original intention and focus
are more inclined to enable users to have a deeper and more comprehensive understanding of their sleep, explore the causes of sleep
quality, and provide users with more practical advice to point them in a clear direction for improving sleep quality and being healthy.
Compared with some medical grade technologies like PSG, our advantages are inexpensive and easy to deploy at home, so it is suitable for
the most general public. Moreover, since it does not need a large number of instruments attached to the user's body, \systemname has less
intrusiveness for sleep and does not require professional personnel to operate. Although the accuracy of {\systemname}'s ability to detect
sleep cannot be compared to medical technology, it is enough for the average family's daily sleep monitoring needs. We want to stress that
our work concentrates on physical activities rather than biomedical signals, so these rich physical activities detected are easily
understood by users, and they can be adjusted with improved and improved sleep based on the results of monitoring.


\begin{table*}[!t]\footnotesize
 \setlength{\tabcolsep}{2.7pt}
\renewcommand{\arraystretch}{0.8}{\multirowsetup}{\centering}
  \caption{Summary of existing solutions.}\label{tab:related_work}
   \vspace{-2mm}
        \begin{tabular}{lcccccc}
        \toprule
        \textbf{System} & \textbf{High accuracy} & \textbf{Practicability} & \textbf{Low disruptive} & \textbf{Low cost} & \textbf{Informativeness} & \textbf{Interpretability}  \\
        \midrule
        \rowcolor{Gray} PSG     &  $\checkmark$ & &  &   & $\checkmark$ &  \\

        Smartphone& &$\checkmark$ &$\checkmark$  &$\checkmark$   & & $\checkmark$ \\

        \rowcolor{Gray} Wearable& &$\checkmark$ & $\checkmark$ & $\checkmark$  & & $\checkmark$ \\
        {\textbf{\systemname}} & &$\checkmark$ &$\checkmark$  & $\checkmark$  & $\checkmark$&$\checkmark$  \\

        \bottomrule
  \end{tabular}
\end{table*}
